\providecommand\classopts{}
\expandafter\documentclass\expandafter[table, usenames, svgnames, dvipsnames,14pt, \classopts]{beamer}
\usetheme[everytitleformat=regular,frametitleformat=lowercase,progressbar=frametitle,numbering=fraction]{m} % load the metropolis theme
\usefonttheme[onlymath]{serif}

\usepackage[portuguese]{babel} % For writing in Portuguese
\usepackage[utf8]{inputenc} % For using unicode characters (easier with accents)
\usepackage{outlines} % For labeled itemize/enumerate blocks
\usepackage{fancybox} % For drawing fancy boxes
\usepackage{ragged2e} % For justifying text
\usepackage{multirow} % For spanning multiple rows in a table
\usepackage{caption}  % For custom captions in tables and figures
\captionsetup{labelformat=empty}

\usepackage{tikz-qtree} % For the graphs
\usetikzlibrary{shapes.arrows,trees,positioning}

\usepackage{stmaryrd} % For some logic symbols
\usepackage{amsmath} % For math symbols
\boldmath

\usepackage{tabularx} % For creating tables with data centered in cells
\newcolumntype{L}[1]{>{\raggedright\let\newline\\\arraybackslash\hspace{0pt}}m{#1}}
\newcolumntype{C}[1]{>{\centering\let\newline\\\arraybackslash\hspace{0pt}}m{#1}}
\newcolumntype{R}[1]{>{\raggedleft\let\newline\\\arraybackslash\hspace{0pt}}m{#1}}

\title{Diagramas de Decisão Binária (BDDs)}
\subtitle{Aula 2}
\date{7 de outubro de 2015}
\author{Luiz Carlos Vieira}
\institute{MAC0239 - Introdução à Lógica e Verificação de Programas}

% definition symbol
\newcommand\defeq{\mathrel{\overset{\makebox[0pt]{\mbox{\normalfont\tiny\sffamily def}}}{=}}}

% Image's path
\DeclareGraphicsExtensions{.pdf,.jpg,.png}
\graphicspath{{./images/}}

\setbeamerfont{footnote}{size=\tiny}

\definecolor{good}{rgb}{0.14,0.67,0.70} %blueish
\definecolor{bad}{rgb}{0.92,0.30,0.36} % redish
\definecolor{ok}{rgb}{0.97,0.61,0.12} % orangeish
\definecolor{emphasis}{rgb}{0.92,0.00,0.70} % pinkish

\begin{document}
\maketitle

%-------------------------------------------
\begin{frame}{Conteúdo}

    \begin{outline}
        \1 BDDs ordenados (OBDDs)

        \vspace{1em}
            
        \1 Algoritmos para OBDDs Reduzidos
            \2[-] algoritmo \texttt{reduzir}
            \2[-] algoritmo \texttt{aplicar}
            \2[-] algoritmo \texttt{restringir}
            \2[-] algoritmo \texttt{existe}
    \end{outline}

\end{frame}
%-------------------------------------------

%-------------------------------------------
\begin{frame}{Relembrando: múltiplas ocorrências}

    \begin{columns}[c]
        \column{0.4\textwidth}

            \begin{figure}
            
                \begin{tikzpicture}
                    [scale=0.8,auto=left,
                        every node/.style={circle, minimum height=4mm, draw=black, thick, align=center, text depth = 0pt, transform shape},
                        every edge/.append style={draw=blue, thick}
                    ]
                    %\draw[help lines] (0,0) grid (8,6);

                    \node (x1) at (4.0, 6.0) {$p$};
                    \node (y1) at (2.5, 4.5) {$q$};
                    \node (z1) at (5.5, 4.5) {$r$};
                    \node (x2) at (1.5, 3.0) {$p$};
                    \node (x3) at (6.5, 3.0) {$p$};
                    \node (y2) at (4.3, 3.0) {$q$};
                    
                    \node[rectangle, scale=0.7] (0) at (2.0, 1.0) {$0$};
                    \node[rectangle, scale=0.7] (1) at (6.0, 1.0) {$1$};

                    \draw[dashed, thick] (x1) to (y1);
                    \draw[thick] (x1) to (z1);
                    \draw[dashed, thick] (y1) to (x2);
                    \draw[thick, bend right=15] (y1) to (1);
                    \draw[thick, emphasis] (x2) to (1);
                    \draw[dashed, thick] (x2) to (0);
                    \draw[thick] (x3) to (1);
                    \draw[dashed, thick] (x3) to (0);
                    \draw[dashed, thick] (z1) to (y2);
                    \draw[thick] (z1) to (x3);
                    \draw[dashed, thick] (y2) to (0);
                    \draw[thick] (y2) to (1);                    

                \end{tikzpicture}

            \end{figure}

        \column{0.6\textwidth}
            
            \begin{outline}
                \small
                \1 A definição de BDDs não impede uma variável de ocorrer mais de uma vez em um caminho
                
                \vspace{1em}
                
                \1 Mas tal representação pode incorrer em desperdícios
                    \2[-] {\scriptsize linha sólida do $p$ à esquerda (colorida) jamais será percorrida}
            \end{outline}
            
            \begin{center}
                \ovalbox{
                    \begin{minipage}[h]{0.9\columnwidth}
                        \noindent\scriptsize\justifying
                        Esse é um resultado comum após as operações discutidas na aula anterior
                    \end{minipage}
                }
            \end{center}            

    \end{columns}
    
\end{frame}
%-------------------------------------------

%-------------------------------------------
\begin{frame}{Relembrando: comparação de \uppercase{BDD}s}

    {\small Além de tornar um BDD menos eficiente, ocorrências múltiplas de uma variável também dificultam a comparação de BDDs}
    
    \begin{columns}[c]
        \column{0.5\textwidth}

            \begin{figure}
            
                \begin{tikzpicture}
                    [scale=0.7,auto=left,
                        every node/.style={circle, minimum height=4mm, draw=black, thick, align=center, text depth = 0pt, transform shape},
                        every edge/.append style={draw=blue, thick}
                    ]
                    %\draw[help lines] (0,0) grid (8,6);
                  
                    \node (x1) at (4.0, 6.0) {$p$};
                    \node (y1) at (2.5, 4.5) {$q$};
                    \node (z1) at (5.5, 4.5) {$r$};
                    \node (x2) at (1.5, 3.0) {$p$};
                    \node (x3) at (6.5, 3.0) {$p$};
                    \node (y2) at (4.3, 3.0) {$q$};
                    
                    \node[rectangle, scale=0.7] (0) at (2.0, 1.0) {$0$};
                    \node[rectangle, scale=0.7] (1) at (6.0, 1.0) {$1$};

                    \draw[dashed, thick] (x1) to (y1);
                    \draw[thick] (x1) to (z1);
                    \draw[dashed, thick] (y1) to (x2);
                    \draw[thick, bend right=15] (y1) to (1);
                    \draw[thick] (x2) to (1);
                    \draw[dashed, thick] (x2) to (0);
                    \draw[thick] (x3) to (1);
                    \draw[dashed, thick] (x3) to (0);
                    \draw[dashed, thick] (z1) to (y2);
                    \draw[thick] (z1) to (x3);
                    \draw[dashed, thick] (y2) to (0);
                    \draw[thick] (y2) to (1);

                \end{tikzpicture}

            \end{figure}

        \column{0.5\textwidth}
            
            \begin{figure}
            
                \begin{tikzpicture}
                    [scale=0.7,auto=left,
                        every node/.style={circle, minimum height=4mm, draw=black, thick, align=center, text depth = 0pt, transform shape},
                        every edge/.append style={draw=blue, thick}
                    ]
                    %\draw[help lines] (0,0) grid (8,6);
                  
                    \node (x) at (4.0, 6.0) {$p$};
                    \node (y1) at (2.5, 4.5) {$q$};
                    \node (y2) at (5.5, 4.5) {$q$};
                    \node (z) at (4.0, 3.0) {$r$};
                    
                    \node[rectangle, scale=0.7] (0) at (2.5, 1.0) {$0$};
                    \node[rectangle, scale=0.7] (1) at (5.5, 1.0) {$1$};

                    \draw[dashed, thick] (x) to (y1);
                    \draw[thick] (x) to (y2);
                    \draw[dashed, thick] (y1) to (0);
                    \draw[thick, bend right] (y1) to (1);
                    \draw[dashed, thick] (y2) to (z);
                    \draw[thick] (y2) to (1);
                    \draw[dashed, thick] (z) to (0);
                    \draw[thick] (z) to (1);
                    
                \end{tikzpicture}

            \end{figure}

    \end{columns}
    
\end{frame}
%-------------------------------------------

%-------------------------------------------
\begin{frame}{Conceito de ordem de um caminho}

    \begin{columns}[c]
        \column{0.5\textwidth}

            \begin{figure}
            
                \begin{minipage}[h]{0.5\columnwidth}
                
                    \begin{outline}
                        \scriptsize
                        \uncover<2->{
                            \1 $[p\uncover<3->{,q}\uncover<4->{,p}]$
                        }
                        \uncover<5->{
                            \1 $[p\uncover<6->{,q}]$
                        }
                        \uncover<7->{
                            \1 $[p\uncover<8->{,r}\uncover<9->{,q}]$
                        }
                        \uncover<10->{
                            \1 $[p\uncover<11->{,r}\uncover<12->{,p}]$
                        }
                    \end{outline}
                    
                \end{minipage}
            
                \vspace{1em}
            
                \begin{tikzpicture}
                    [scale=0.5,auto=left,
                        every node/.style={circle, minimum height=4mm, draw=black, thick, align=center, text depth = 0pt, transform shape},
                        every edge/.append style={draw=blue, thick}
                    ]
                    %\draw[help lines] (0,0) grid (8,6);
                  
                    \node (p1) at (4.0, 6.0) {$p$};
                    \node (q1) at (2.5, 4.5) {$q$};
                    \node (r1) at (5.5, 4.5) {$r$};
                    \node (p2) at (1.5, 3.0) {$p$};
                    \node (p3) at (6.5, 3.0) {$p$};
                    \node (q2) at (4.3, 3.0) {$q$};
                    
                    \node[rectangle, scale=0.7] (0) at (2.0, 1.0) {$0$};
                    \node[rectangle, scale=0.7] (1) at (6.0, 1.0) {$1$};

                    \draw[dashed, thick] (p1) to (q1);
                    \draw[thick] (p1) to (r1);
                    \draw[dashed, thick] (q1) to (p2);
                    \draw[thick, bend right=15] (q1) to (1);
                    \draw[thick] (p2) to (1);
                    \draw[dashed, thick] (p2) to (0);
                    \draw[thick] (p3) to (1);
                    \draw[dashed, thick] (p3) to (0);
                    \draw[dashed, thick] (r1) to (q2);
                    \draw[thick] (r1) to (p3);
                    \draw[dashed, thick] (q2) to (0);
                    \draw[thick] (q2) to (1);

                    \uncover<2-4>{
                        \node[emphasis] (p1) at (4.0, 6.0) {$p$};
                    }
                    \uncover<3-4>{
                        \node[emphasis] (q1) at (2.5, 4.5) {$q$};
                        \draw[dashed, thick, emphasis] (p1) to (q1);
                    }
                    \uncover<4-4>{
                        \node[emphasis] (p2) at (1.5, 3.0) {$p$};
                        \draw[dashed, thick, emphasis] (q1) to (p2);
                        \draw[thick, emphasis] (p2) to (1);
                        \draw[dashed, thick, emphasis] (p2) to (0);
                    }
                    
                    \uncover<5-6>{
                        \node[emphasis] (p1) at (4.0, 6.0) {$p$};
                    }
                    \uncover<6-6>{
                        \node[emphasis] (q1) at (2.5, 4.5) {$q$};
                        \draw[dashed, thick, emphasis] (p1) to (q1);
                        \draw[thick, bend right=15, emphasis] (q1) to (1);
                    }
                    
                    \uncover<7-9>{
                        \node[emphasis] (p1) at (4.0, 6.0) {$p$};
                    }
                    \uncover<8-9>{
                        \node[emphasis] (r1) at (5.5, 4.5) {$r$};
                        \draw[thick, emphasis] (p1) to (r1);
                    }
                    \uncover<9-9>{
                        \node[emphasis] (q2) at (4.3, 3.0) {$q$};
                        \draw[dashed, thick, emphasis] (r1) to (q2);
                        \draw[dashed, thick, emphasis] (q2) to (0);
                        \draw[thick, emphasis] (q2) to (1);
                    }

                    \uncover<10-12>{
                        \node[emphasis] (p1) at (4.0, 6.0) {$p$};
                    }
                    \uncover<11-12>{
                        \node[emphasis] (r1) at (5.5, 4.5) {$r$};
                        \draw[thick, emphasis] (p1) to (r1);
                    }
                    \uncover<12-12>{
                        \node[emphasis] (p3) at (6.5, 3.0) {$p$};
                        \draw[thick, emphasis] (r1) to (p3);
                        \draw[thick, emphasis] (p3) to (1);
                        \draw[dashed, thick, emphasis] (p3) to (0);
                    }
                \end{tikzpicture}

            \end{figure}
            
            \only<20>{
                \ovalbox{
                    \begin{minipage}[h]{0.8\columnwidth}
                        \noindent\scriptsize
                        Ocorrem repetições e não há um padrão na ordenação
                    \end{minipage}
                }            
            }

        \column{0.5\textwidth}
            
            \begin{figure}
            
                \begin{minipage}[h]{0.5\columnwidth}
                
                    \begin{outline}
                        \scriptsize
                        \uncover<13->{
                            \1 $[p\uncover<14->{,q}]$
                        }
                        \uncover<15->{
                            \1 $[p\uncover<16->{,q}\uncover<17->{,r}]$
                        }
                        \uncover<18->{
                            \1 $[p\uncover<19->{,q}]$
                        }
                    \end{outline}
                    
                \end{minipage}
            
                \vspace{1em}
            
                \begin{tikzpicture}
                    [scale=0.5,auto=left,
                        every node/.style={circle, minimum height=4mm, draw=black, thick, align=center, text depth = 0pt, transform shape},
                        every edge/.append style={draw=blue, thick}
                    ]
                    %\draw[help lines] (0,0) grid (8,6);
                  
                    \node (p) at (4.0, 6.0) {$p$};
                    \node (q1) at (2.5, 4.5) {$q$};
                    \node (q2) at (5.5, 4.5) {$q$};
                    \node (r) at (4.0, 3.0) {$r$};
                    
                    \node[rectangle, scale=0.7] (0) at (2.5, 1.0) {$0$};
                    \node[rectangle, scale=0.7] (1) at (5.5, 1.0) {$1$};

                    \draw[dashed, thick] (p) to (q1);
                    \draw[thick] (p) to (q2);
                    \draw[dashed, thick] (q1) to (0);
                    \draw[thick, bend right] (q1) to (1);
                    \draw[dashed, thick] (q2) to (r);
                    \draw[thick] (q2) to (1);
                    \draw[dashed, thick] (r) to (0);
                    \draw[thick] (r) to (1);
                    
                    \uncover<13-14>{
                        \node[emphasis] (p) at (4.0, 6.0) {$p$};
                    }
                    \uncover<14-14>{
                        \node[emphasis] (q1) at (2.5, 4.5) {$q$};
                        \draw[dashed, thick, emphasis] (p) to (q1);
                        \draw[dashed, thick, emphasis] (q1) to (0);
                        \draw[thick, bend right, emphasis] (q1) to (1);
                    }
                    
                    \uncover<15-17>{
                        \node[emphasis] (p) at (4.0, 6.0) {$p$};
                    }
                    \uncover<16-17>{
                        \node[emphasis] (q2) at (5.5, 4.5) {$q$};
                        \draw[thick, emphasis] (p) to (q2);
                    }
                    \uncover<17-17>{
                        \node[emphasis] (r) at (4.0, 3.0) {$r$};
                        \draw[dashed, thick, emphasis] (q2) to (r);
                        \draw[dashed, thick, emphasis] (r) to (0);
                        \draw[thick, emphasis] (r) to (1);
                    }
                    
                    \uncover<18-18>{
                        \node[emphasis] (p) at (4.0, 6.0) {$p$};
                    }
                    \uncover<19-19>{
                        \node[emphasis] (q2) at (5.5, 4.5) {$q$};
                        \draw[thick, emphasis] (q2) to (1);
                    }
                    
                \end{tikzpicture}

            \end{figure}

            \only<20>{
                \ovalbox{
                    \begin{minipage}[h]{0.8\columnwidth}
                        \noindent\scriptsize
                        Não ocorrem repetições e há um padrão na ordenação
                    \end{minipage}
                }            
            }
            
    \end{columns}
    
\end{frame}
%-------------------------------------------

%-------------------------------------------
\begin{frame}{Ordenação de \uppercase{BDD}s}

    \begin{outline}
        \1 Se a ordem das variáveis em qualquer caminho for sempre a mesma, a comparação é trivial
            \2[-] basta verificar se os BDDs têm a mesma estrutura

        \vspace{1em}
            
        \1 Quando a ordem das variáveis de teste é sempre a mesma, o BDD é dito \textit{ordenado}
            \2[-] e passa a ser chamado Diagrama de Busca Binária Ordenado (OBDD)
    \end{outline}

\end{frame}
%-------------------------------------------

%-------------------------------------------
\begin{frame}{Definição: \uppercase{OBDD}s}

    \begin{block}{\textbf{Definição 6.6}}
        Seja $[p_1,p_2,...,p_n]$ uma lista ordenada de variáveis sem duplicação e seja $B$ um BDD tal que todas as suas variáveis aparecem em algum lugar da lista. Dizemos que $B$ tem a ordem $[p_1,p_2,...,p_n]$ se todos os nós de variáveis de $B$ ocorrem na lista, e, para toda ocorrência de $p_i$ seguido de $p_j$ ao longo de qualquer caminho em $B$ temos $i < j$.
    \end{block}

\end{frame}
%-------------------------------------------

%-------------------------------------------
\begin{frame}{Exemplo de \uppercase{BDD} ordenado}

    \begin{figure}
    
        \caption{Ordem: $[p, q, r]$}
    
        \begin{tikzpicture}
            [scale=1.0,auto=left,
                every node/.style={circle, minimum height=4mm, draw=black, thick, align=center, text depth = 0pt, transform shape},
                every edge/.append style={draw=blue, thick}
            ]
            %\draw[help lines] (0,0) grid (8,6);

            \node (p) at (3.0, 5.5)  {$p$};
            \node (q1) at (1.5, 4.0)  {$q$};
            \node (q2) at (4.5, 4.0)  {$q$};
            
            \node (r1) at (0.6, 2.5)  {$r$};
            \node (r2) at (2.4, 2.5)  {$r$};
            \node (r3) at (3.6, 2.5)  {$r$};
            \node (r4) at (5.4, 2.5)  {$r$};
            
            \node[rectangle, scale=0.7] (0) at (1.5, 0.5) {$0$};
            \node[rectangle, scale=0.7] (1) at (4.5, 0.5) {$1$};

            \draw[dashed, thick] (p) to (q1);
            \draw[thick] (p) to (q2);
            \draw[dashed, thick] (q1) to (r1);
            \draw[thick] (q1) to (r2);
            \draw[dashed, thick] (q2) to (r3);
            \draw[thick] (q2) to (r4);
            
            \draw[dashed, thick] (r1) to (0);
            \draw[thick] (r1) to (1);
            \draw[dashed, thick] (r2) to (0);
            \draw[thick] (r2) to (1);
            \draw[dashed, thick] (r3) to (0);
            \draw[thick] (r3) to (1);
            \draw[thick] (r4) to (0);
            \draw[dashed, thick] (r4) to (1);            
        \end{tikzpicture}

    \end{figure}

\end{frame}
%-------------------------------------------

%-------------------------------------------
\begin{frame}{Outro exemplo de \uppercase{BDD} ordenado}

    \begin{figure}
    
        \caption{Ordem: $[p, q, r]$}
    
        \begin{tikzpicture}
            [scale=1.0,auto=left,
                every node/.style={circle, minimum height=4mm, draw=black, thick, align=center, text depth = 0pt, transform shape},
                every edge/.append style={draw=blue, thick}
            ]
            %\draw[help lines] (0,0) grid (8,6);
          
            \node (x) at (4.0, 6.0) {$p$};
            \node (y1) at (2.5, 4.5) {$q$};
            \node (y2) at (5.5, 4.5) {$q$};
            \node (z) at (4.0, 3.0) {$r$};
            
            \node[rectangle, scale=0.7] (0) at (2.5, 1.0) {$0$};
            \node[rectangle, scale=0.7] (1) at (5.5, 1.0) {$1$};

            \draw[dashed, thick] (x) to (y1);
            \draw[thick] (x) to (y2);
            \draw[dashed, thick] (y1) to (0);
            \draw[thick, bend right] (y1) to (1);
            \draw[dashed, thick] (y2) to (z);
            \draw[thick] (y2) to (1);
            \draw[dashed, thick] (z) to (0);
            \draw[thick] (z) to (1);
            
        \end{tikzpicture}

    \end{figure}

\end{frame}
%-------------------------------------------
    
%-------------------------------------------
\begin{frame}{Exemplo de \uppercase{BDD} não ordenado}

    \begin{figure}
    
        \caption{Sem ordem definida ($[p, q, r]$ à esquerda e $[p, r, q]$ à direita)}
    
        \begin{tikzpicture}
            [scale=1.0,auto=left,
                every node/.style={circle, minimum height=4mm, draw=black, thick, align=center, text depth = 0pt, transform shape},
                every edge/.append style={draw=blue, thick}
            ]
            %\draw[help lines] (0,0) grid (8,6);

            \node (p) at (3.0, 5.5)  {$p$};
            \node (q1) at (1.5, 4.0)  {$q$};
            \node (r1) at (4.5, 4.0)  {$r$};
            
            \node (r2) at (0.6, 2.5)  {$r$};
            \node (q2) at (5.4, 2.5)  {$q$};
            
            \node[rectangle, scale=0.7] (0) at (1.5, 0.5) {$0$};
            \node[rectangle, scale=0.7] (1) at (4.5, 0.5) {$1$};

            \draw[dashed, thick] (p) to (q1);
            \draw[thick] (p) to (r1);
            \draw[dashed, thick] (q1) to (r2);
            \draw[thick] (q1) to (1);
            \draw[dashed, thick] (r1) to (1);
            \draw[thick] (r1) to (q2);
            \draw[dashed, thick] (r2) to (0);
            \draw[thick] (r2) to (1);
            \draw[dashed, thick] (q2) to (0);
            \draw[thick] (q2) to (1);
        \end{tikzpicture}

    \end{figure}

\end{frame}
%-------------------------------------------

%-------------------------------------------
\begin{frame}{Vantagens da ordenação de \uppercase{BDD}s}

    \begin{outline}
        \1 A comparação de dois BDDs de ordens compatíveis é imediata
        
        \vspace{1em}
        
        \1 Aplicações das reduções C1-C3 em um OBDD garantidamente mantêm sua ordem original
        
        \vspace{1em}
        
        \1 Esse compromisso com a ordem produz uma representação única de funções booleanas com OBDDs reduzidos
            \2[-] chamada de \textit{forma canônica}
    \end{outline}

\end{frame}
%-------------------------------------------

%-------------------------------------------
\begin{frame}{Teorema: \uppercase{OBDD}s reduzidos são únicos}

    \begin{block}{\textbf{Teorema 6.7}}
        A representação em OBDD reduzido de uma função dada $\phi$ é unica. Isto é, sejam $B$ e $B^\prime$ dois OBDDs reduzidos com ordens compatíveis. Se $B$ e $B^\prime$ representam a mesma função booleana, então eles têm estruturas idênticas.
    \end{block}

\end{frame}
%-------------------------------------------

%-------------------------------------------
\begin{frame}{Características de \uppercase{OBDD}s}

    \begin{outline}
        \1 As simplificações C1-C3 em um OBDD produzem sempre o mesmo OBDD reduzido
            \2[-] chamado então de \textit{forma canônica}
        
        \vspace{1em}
        
        \1 ODDBs permitem representações compactas de certas classes de funções booleanas
            \2[-] que seriam exponenciais em outros formatos/representações
            
        \vspace{1em}
        
        \1 Por outro lado, as operações $\land$ e $\lor$ apresentadas anteriormente não funcionam
            \2[-] pois podem introduzir ocorrências múltiplas de uma mesma variável
    \end{outline}

\end{frame}
%-------------------------------------------

%-------------------------------------------
\begin{frame}{Impacto da escolha da ordenação}


\end{frame}
%-------------------------------------------


%-------------------------------------------
\begin{frame}{Importância da representação canônica}

    \scriptsize

    \begin{outline}
        \1 \textbf{Ausência de variáveis redundantes}. Se o valor de uma função booleana não depende de uma variável, então nenhum OBDD reduzido que a represente contém tal variável;
        
        \vspace{1em}
        
        \1 \textbf{Teste de equivalência semântica}. Se duas funções são representadas por OBDD com ordem compatível, é possível decidir eficientemente se são equivalentes reduzindo seus OBDD e comparando sua estrutura;
            
        \vspace{1em}
        
        \1 \textbf{Teste de validade}. Se uma função booleana é válida, seu OBDD reduzido é igual a $B_1$;
        
        \vspace{1em}
        
        \1 \textbf{Teste de implicação}. Pode-se testar se uma função $\phi$ implica em outra $\psi$ calculando o OBDD para $\phi \land \psi$ e verificando que ele é igual a $B_0$;
        
        \vspace{1em}
        
        \1 \textbf{Teste de satisfação}. Se uma função booleana é satisfeita, então seu OBDD reduzido não é igual a $B_0$.
    \end{outline}

\end{frame}
%-------------------------------------------

%-------------------------------------------
%\begin{frame}{Expansão de Shannon}


%\end{frame}
%-------------------------------------------

%-------------------------------------------
%\begin{frame}{Algoritmo \texttt{reduzir}}


%\end{frame}
%-------------------------------------------

%-------------------------------------------
%\begin{frame}{Algoritmo \texttt{aplicar}}


%\end{frame}
%-------------------------------------------

%-------------------------------------------
%\begin{frame}{Algoritmo \texttt{restringir}}


%\end{frame}
%-------------------------------------------

%-------------------------------------------
%\begin{frame}{Algoritmo \texttt{existe}}


%\end{frame}
%-------------------------------------------

\end{document}