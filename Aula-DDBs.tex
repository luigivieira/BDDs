\providecommand\classopts{}
\expandafter\documentclass\expandafter[table, usenames, svgnames, dvipsnames,14pt, \classopts]{beamer}
\usetheme[everytitleformat=regular,frametitleformat=lowercase,progressbar=frametitle,numbering=fraction]{m} % load the metropolis theme
\usefonttheme[onlymath]{serif}

\usepackage[portuguese]{babel} % For writing in Portuguese
\usepackage[utf8]{inputenc} % For using unicode characters (easier with accents)
\usepackage{outlines} % For labeled itemize/enumerate blocks
\usepackage{fancybox} % For drawing fancy boxes
\usepackage{ragged2e} % For justifying text
\usepackage{multirow} % For spanning multiple rows in a table
\usepackage{caption}  % For custom captions in tables and figures
\captionsetup{labelformat=empty}

\usepackage{tikz-qtree} % For the graphs
\usetikzlibrary{shapes.arrows,trees,positioning}

\usepackage{amsmath} % For math symbols
\boldmath

\usepackage{tabularx} % For creating tables with data centered in cells
\newcolumntype{L}[1]{>{\raggedright\let\newline\\\arraybackslash\hspace{0pt}}m{#1}}
\newcolumntype{C}[1]{>{\centering\let\newline\\\arraybackslash\hspace{0pt}}m{#1}}
\newcolumntype{R}[1]{>{\raggedleft\let\newline\\\arraybackslash\hspace{0pt}}m{#1}}

\title{Diagramas de Decisão Binários (BDDs)}
\date{\today}
\author{Luiz Carlos Vieira}
\institute{Instituto de Matemática e Estatística da Universidade de São Paulo}

% definition symbol
\newcommand\defeq{\mathrel{\overset{\makebox[0pt]{\mbox{\normalfont\tiny\sffamily def}}}{=}}}

% Image's path
\DeclareGraphicsExtensions{.pdf,.jpg,.png}
\graphicspath{{./images/}}

\setbeamerfont{footnote}{size=\tiny}

\begin{document}
\maketitle

%-------------------------------------------
\begin{frame}{Conteúdo}

    \begin{outline}
        \1 Representação de Funções Booleanas
            \2[-] fórmulas proposicionais e tabelas-verdade
            \2[-] diagramas de decisão binários (BDDs)
            \2[-] diagramas de decisão binários ordenados (OBDDs)

        \vspace{1em}
            
        \1 Algoritmos para OBDDs Reduzidos
            \2[-] algoritmo \texttt{reduzir}
            \2[-] algoritmo \texttt{aplicar}
            \2[-] algoritmo \texttt{restringir}
            \2[-] algoritmo \texttt{existe}
    \end{outline}

\end{frame}
%-------------------------------------------

%===========================================
\section{Representação de Funções Booleanas}
%===========================================

%-------------------------------------------
\begin{frame}{Funções booleanas}

    \begin{outline}
        \1 Parte fundamental do formalismo descritivo de sistemas de \textit{hardware} e \textit{software}

        \vspace{1em}
            
        \1 Que precisa ser computacionalmente representado de forma eficiente
    \end{outline}

\end{frame}
%-------------------------------------------

%-------------------------------------------
\begin{frame}{Definição: variáveis booleanas}

    \begin{block}{\textbf{Definição 6.1(a)}}
        Uma variável booleana $x$ é uma variável que só pode assumir os valores $0$ e $1$. Denotamos variáveis booleanas por $x_1$, $x_2$, $...$, e $x$, $y$ e $z$, $...$.
    \end{block}

\end{frame}
%-------------------------------------------

%-------------------------------------------
\begin{frame}{Definição: funções booleanas}

    \begin{block}{\textbf{Definição 6.1(b)}}
        As seguintes funções são definidas no conjunto $\{0,1\}$:
        \\
        \begin{outline}
            \1 $\overline{0} \defeq 1$ e $\overline{1} \defeq 0$;
                
            \1 $x \cdot y \defeq 1$ se $x$ e $y$ têm valor $1$; caso contrário, $x \cdot y \defeq 0$;
            
            \1 $x + y \defeq 0$ se $x$ e $y$ têm valor $0$; caso contrário, $x + y \defeq 1$;
            
            \1 $x \oplus y \defeq 1$ se exatamente um entre $x$ e $y$ é igual a $1$; caso contrário, $x \oplus y \defeq 0$.
        \end{outline}        
        
    \end{block}

\end{frame}
%-------------------------------------------

%-------------------------------------------
\begin{frame}{Funções e variáveis booleanas}

    \begin{outline}
        \1 Uma função booleana $f$ com $n$ variáveis é uma função de $\{0,1\}^n$ para $\{0,1\}$.
        
        \vspace{1em}
        
        \1 Escreve-se $f(x_1,x_2,\dots,x_n)$ ou $f(\mathcal{V})$ para indicar que uma representação sintática de $f$ só depende das variáveis booleanas em $\mathcal{V}$.
    \end{outline}
    
\end{frame}
%-------------------------------------------

%-------------------------------------------
\begin{frame}{Alguns exemplos de funções booleanas}

    \begin{outline}[enumerate]
        \1 $f(x,y) \defeq x \cdot (y + \overline{x})$
            
        \vspace{1em}
            
        \1 $g(x,y) \defeq x \cdot y + (1 \oplus \overline{x})$

        \vspace{1em}

        \1 $h(x,y,z) \defeq x + y \cdot (x \oplus \overline{y})$

        \vspace{1em}

        \1 $k() \defeq 1 \oplus (0 \cdot \overline{1})$
    \end{outline}        

\end{frame}
%-------------------------------------------

%-------------------------------------------
\begin{frame}{wffs e tabelas-verdade}

    As fórmulas proposicionais bem-formadas (\textit{wffs}) e as tabelas-verdade são duas representações de funções booleanas

    \begin{outline}
        \1 \textbf{fórmulas proposicionais}:
            \2[-] $\land$ denota $\cdot$
            \2[-] $\lor$ denota $+$
            \2[-] $\lnot$ denota $\bar{~}$
            \2[-] e $\top$ e $\bot$ denotam, respectivamente, $1$ e $0$
        
        \vspace{1em}
        
        \1 \textbf{tabelas-verdade}: representam funções booleanas de maneira óbvia
    \end{outline}        
            
\end{frame}
%-------------------------------------------

%-------------------------------------------
\begin{frame}{Tabelas-verdade de funções booleanas}

    \begin{columns}[c]
        \column{0.5\textwidth}
            \begin{center}
                \begin{table}
                    \caption{Tabela-verdade da função booleana $f(x,y) \defeq \overline{x + y}$}
                    \vspace{1em}
                    \begin{tabular}{cc|c}
                        $x$ & $y$ & $f(x,y)$\\
                        \hline
                        $1$ & $1$ & $0$\\
                        $0$ & $1$ & $0$\\
                        $1$ & $0$ & $0$\\
                        $0$ & $0$ & $1$\\
                    \end{tabular}
                \end{table}
            \end{center}

        \column{0.5\textwidth}
            \begin{center}
                \begin{table}
                    \caption{Tabela-verdade da fórmula proposicional $\phi \equiv \lnot(p \lor q)$}
                    \vspace{1em}
                    \begin{tabular}{cc|c}
                        $p$ & $q$ & $\phi$\\
                        \hline
                        $V$ & $V$ & $F$\\
                        $F$ & $V$ & $F$\\
                        $V$ & $F$ & $F$\\
                        $F$ & $F$ & $V$\\
                    \end{tabular}
                \end{table}
            \end{center}
    \end{columns}
            
\end{frame}
%-------------------------------------------

%-------------------------------------------
\begin{frame}{Vantagens e desvantagens}

    {\small
        Há vantagens e desvantagens no uso de tabelas-verdade e fórmulas proposicionais para representar funções booleanas
    }

    {\scriptsize
        \setlength{\tabcolsep}{2pt}
        \renewcommand{\arraystretch}{1.5}        
        \begin{table}
            \begin{tabular}{C{0.3\textwidth}|C{0.35\textwidth}C{0.35\textwidth}}
                & {\small\textbf{Tabelas-Verdade}} & {\small\textbf{Fórmulas Proposicionais}}\\
                \hline
                {\small\textbf{Vantagens}} & verificações\footnote{satisfação, validade e equivalência\label{verificacoes}} simples & representação compacta\\

                {\small\textbf{Desvantagens}} & ineficientes em espaço & verificações\footnotemark[1] não tão simples\\
            \end{tabular}
        \end{table}
    }

    \vspace{-2em}
    
    \begin{center}
        \ovalbox{
            \begin{minipage}[h]{0.7\linewidth}
                \noindent\scriptsize\centering
                Ambas são computacionalmente caras para muitas variáveis
            \end{minipage}
        }
    \end{center}

\end{frame}
%-------------------------------------------

%-------------------------------------------
\begin{frame}{Também nas operações booleanas}

    As operações booleanas ($\cdot$, $+$, $\oplus$ e $\bar{~}$) entre duas funções $f$ e $g$ também são simples:

    \begin{outline}
        \1 Com tabelas-verdade
            \2[-] operação diretamente aplicada a cada linha
            \2[-] acrescentando variáveis inexistentes, se necessário

        \vspace{1em}
            
        \1 Com fórmulas proposicionais
            \2[-] manipulação sintática da Lógica Proposicional
    \end{outline}

    \begin{center}
        \ovalbox{
            \begin{minipage}[h]{0.9\linewidth}
                \noindent\scriptsize\justifying
                Computacionalmente caro com tabelas-verdade ($2^n$ linhas) e imediata com fórmulas proposicionais (por exemplo, $f \cdot g$ e $f \oplus g$ são respectivamente $\phi \land \psi$ e $(\phi \land \lnot{\psi}) \lor (\lnot{\phi} \land \psi)$)
            \end{minipage}
        }
    \end{center}
    
\end{frame}
%-------------------------------------------

%-------------------------------------------
\begin{frame}{Utilizando formas normais}

    \begin{outline}
        \1 A representação de fórmulas proposicionais em formas normais é facilitada em alguns aspectos
            \2[-] mas é dificultada em outros
        
        \vspace{1em}
        
        \1 De forma geral, elas podem ser muito longas no pior caso
    \end{outline}
\end{frame}
%-------------------------------------------

%-------------------------------------------
\begin{frame}{Forma Normal Conjuntiva (\uppercase{CNF})}

    \begin{outline}
        \1 Facilita o teste de validade
            \2[-] cláusula disjuntiva sem preposições complementares
            \2[-] teste de satisfação não é semelhante
        
        \vspace{1em}
        
        \1 Facilita a operação de conjunção ($\land$)
            \2[-] se $\phi$ e $\psi$ são CNFs, o resultado de $\phi \land \psi$ é CNF
        
        \vspace{1em}
        
        \1 Dificulta as demais operações ($\lor$ e $\lnot$)
            \2[-] aplicação de distributividade para manter CNF
    \end{outline}

    \begin{center}
        \ovalbox{
            \begin{minipage}[h]{0.8\linewidth}
                \noindent\scriptsize
                A forma normal disjuntiva (DNF) -- disjunção de conjunções -- é dual com a CNF em relação a essas propriedades
            \end{minipage}
        }
    \end{center}
    
\end{frame}
%-------------------------------------------

%-------------------------------------------
\begin{frame}{Resumo da eficiência das representações}

    \begin{center}
        \scriptsize
        
        \setlength{\tabcolsep}{1pt}
        \renewcommand{\arraystretch}{1.5}        
        
        \begin{table}
            \begin{tabular}{C{0.25\textwidth}|C{0.15\textwidth}*{5}{C{0.12\textwidth}}}
                & & \multicolumn{2}{c}{\textbf{teste de}} & \multicolumn{3}{c}{\textbf{operações booleanas}}\\[-3.5mm]
                \textbf{Representação de funções booleanas} & \textbf{compacta?} & \textbf{satisfação} & \textbf{validade} & $\cdot$ & $+$ & $\bar{~}$\\
                \hline
                \textbf{fórmulas proposicionais} & \textcolor{ForestGreen}{muitas vezes} & \textcolor{BrickRed}{difícil} & \textcolor{BrickRed}{difícil} & \textcolor{ForestGreen}{fácil} & \textcolor{ForestGreen}{fácil} & \textcolor{ForestGreen}{fácil} \\
                \textbf{fórmulas CNF} & \textcolor{teal}{algumas vezes} & \textcolor{BrickRed}{difícil} & \textcolor{ForestGreen}{fácil} & \textcolor{ForestGreen}{fácil} & \textcolor{BrickRed}{difícil} & \textcolor{BrickRed}{difícil}\\
                \textbf{fórmulas NDF} & \textcolor{teal}{algumas vezes} & \textcolor{ForestGreen}{fácil} & \textcolor{BrickRed}{difícil} & \textcolor{BrickRed}{difícil} & \textcolor{ForestGreen}{fácil} & \textcolor{BrickRed}{difícil}\\
                \textbf{tabelas-verdade ordenadas} & \textcolor{BrickRed}{nunca} & \textcolor{BrickRed}{difícil} & \textcolor{BrickRed}{difícil} & \textcolor{BrickRed}{difícil} & \textcolor{BrickRed}{difícil} & \textcolor{BrickRed}{difícil}\\
                \textbf{OBDDs\footnote{Diagramas de Decisão Binários Ordenados -- que serão explorados a seguir} reduzidos} & \textcolor{ForestGreen}{muitas vezes} & \textcolor{ForestGreen}{fácil} & \textcolor{ForestGreen}{fácil} & \textcolor{teal}{mais ou menos} & \textcolor{teal}{mais ou menos} & \textcolor{ForestGreen}{fácil}\\
            \end{tabular}
        \end{table}
    \end{center}
\end{frame}
%-------------------------------------------

%-------------------------------------------
\begin{frame}{Definição: Árvore de Decisão Binária Finita}

    \begin{block}{\textbf{Definição 6.3}}
        Seja $T$ uma árvore de decisão binária finita. Então $T$ determina \textit{uma única} função booleana das variáveis nos nós não-terminais da seguinte maneira:
        
        \begin{quotation}
        \noindent\small\justifying
        Dada uma atribuição de $0$'s e $1$'s às variáveis booleanas que ocorrem em $T$, começamos pela raiz de $T$ e pegamos a \textit{linha tracejada} sempre que o valor da variável no nó atual é $0$; caso contrário, percorremos a \textit{linha sólida}. O valor da função é o valor do nó terminal atingido.
        \end{quotation}
    \end{block}

\end{frame}
%-------------------------------------------

%-------------------------------------------
\begin{frame}{Por exemplo}

    \begin{columns}[c]
        \column{0.4\textwidth}

            \begin{figure}

                \begin{tikzpicture}
                    [scale=1.0,auto=left,
                        every node/.style={circle, minimum height=4mm, draw=black, thick, align=center, text depth = 0pt, transform shape},
                        every edge/.append style={draw=blue, thick}
                    ]
                    %\draw[help lines] (0,0) grid (6,6);
                  
                    \node (x) at (3.0, 5.0)  {$x$};
                    \node (y1) at (1.5, 3.0)  {$y$};
                    \node (y2) at (4.5, 3.0)  {$y$};
                    \node[rectangle, scale=0.7] (1) at (0.5, 1.0) {$1$};
                    \node[rectangle, scale=0.7] (0a) at (2.5, 1.0) {$0$};
                    \node[rectangle, scale=0.7] (0b) at (3.5, 1.0) {$0$};
                    \node[rectangle, scale=0.7] (0c) at (5.5, 1.0) {$0$};

                    \draw[dashed, thick] (x) to (y1);
                    \draw[thick] (x) to (y2);
                    \draw[dashed, thick] (y1) to (1);
                    \draw[thick] (y1) to (0a);
                    \draw[dashed, thick] (y2) to (0b);
                    \draw[thick] (y2) to (0c);

                    % Animations
                    \uncover<3->{
                        \node[magenta] at (3.0, 5.0)  {$x$};
                    }
                    \uncover<4->{
                        \draw[dashed, thick, magenta] (x) to (y1);
                    }
                    \uncover<5->{
                        \node[magenta] at (1.5, 3.0)  {$y$};
                        \draw[thick, magenta] (y1) to (0a);
                    }
                    \uncover<6->{
                        \node[rectangle, scale=0.7, magenta] at (2.5, 1.0)  {$0$};
                    }                  
                \end{tikzpicture}

            \end{figure}

        \column{0.6\textwidth}
            \begin{outline}
                \renewcommand{\outlineii}{enumerate}
                \1 Árvore da função: $f(x,y) \defeq \overline{x + y}$
               
                \vspace{1em}
               
                \uncover<2->{\1 Para encontrar $f(0,1)$:}
                    \uncover<3->{\2 \textcolor<2>{red}{inicia-se pela raiz}}
                    \uncover<4->{\2 \textcolor<3>{red}{como $x$ é $0$, segue-se pela linha pontilhada}}
                    \uncover<5->{\2 \textcolor<4>{red}{como $y$ é $1$, segue-se pela linha sólida}}
                    \uncover<6->{\2 \textcolor<5>{red}{chega-se à folha $0$; logo $f(0,1) = 0$}}
            \end{outline}
    
    \end{columns}
            
\end{frame}
%-------------------------------------------

%-------------------------------------------
\begin{frame}{Comparando com a tabela-verdade}

    Para a função booleana $f(x,y) \defeq \overline{x + y}$:

    \vspace{-0.5em}
    
    \begin{center}
        \scriptsize
        \ovalbox{
            \begin{minipage}[h]{0.8\linewidth}
                \centering
                equivalente à fórmula proposicional $\phi \equiv \lnot{(p \lor q)}$
            \end{minipage}
        }
    \end{center}    
    
    \vspace{-1.0em}
    
    \begin{columns}[c]
        \column{0.6\textwidth}

            \begin{figure}

                \begin{tikzpicture}
                    [scale=1.0,auto=left,
                        every node/.style={circle, minimum height=4mm, draw=black, thick, align=center, text depth = 0pt, transform shape},
                        every edge/.append style={draw=blue, thick}
                    ]
                    %\draw[help lines] (0,0) grid (6,6);
                  
                    \node (x) at (3.0, 5.0)  {$x$};
                    \node (y1) at (1.5, 3.0)  {$y$};
                    \node (y2) at (4.5, 3.0)  {$y$};
                    \node[rectangle, scale=0.7] (1) at (0.5, 1.0) {$1$};
                    \node[rectangle, scale=0.7] (0a) at (2.5, 1.0) {$0$};
                    \node[rectangle, scale=0.7] (0b) at (3.5, 1.0) {$0$};
                    \node[rectangle, scale=0.7] (0c) at (5.5, 1.0) {$0$};

                    \draw[dashed, thick] (x) to (y1);
                    \draw[thick] (x) to (y2);
                    \draw[dashed, thick] (y1) to (1);
                    \draw[thick] (y1) to (0a);
                    \draw[dashed, thick] (y2) to (0b);
                    \draw[thick] (y2) to (0c);
                \end{tikzpicture}

            \end{figure}

        \column{0.4\textwidth}
            \begin{center}
                \begin{table}
                    \begin{tabular}{cc|c}
                        $x$ & $y$ & $f(x,y)$\\
                        \hline
                        $0$ & $0$ & $1$\\
                        $0$ & $1$ & $0$\\
                        $1$ & $0$ & $0$\\
                        $1$ & $1$ & $0$\\
                    \end{tabular}
                \end{table}
            \end{center}
    
    \end{columns}
            
\end{frame}
%-------------------------------------------

%-------------------------------------------
\begin{frame}{Outro exemplo comparativo}

    Para a função booleana $f(x,y,z) \defeq \overline{x} + (y \cdot z)$:
    
    \vspace{-0.5em}
    
    \begin{center}
        \scriptsize
        \ovalbox{
            \begin{minipage}[h]{0.8\linewidth}
                \centering
                equivalente à fórmula proposicional $\phi \equiv p \to (q \land r)$
            \end{minipage}
        }
    \end{center}
   
   \vspace{-1.0em}
   
    \begin{columns}[c]
        \column{0.6\textwidth}

            \begin{figure}

                \begin{tikzpicture}
                    [scale=0.7,auto=left,
                        every node/.style={circle, minimum height=4mm, draw=black, thick, align=center, text depth = 0pt, transform shape},
                        every edge/.append style={draw=blue, thick}
                    ]
                    %\draw[help lines] (0,0) grid (8,6);
                  
                    \node (x) at (4.0, 6.0)  {$x$};
                    \node (y1) at (2.0, 4.0)  {$y$};
                    \node (y2) at (6.0, 4.0)  {$y$};
                    \node (z1) at (1.0, 2.0)  {$z$};
                    \node (z2) at (3.0, 2.0)  {$z$};
                    \node (z3) at (5.0, 2.0)  {$z$};
                    \node (z4) at (7.0, 2.0)  {$z$};
                    
                    \node[rectangle, scale=0.7] (1a) at (0.4, 0.0) {$1$};
                    \node[rectangle, scale=0.7] (1b) at (1.6, 0.0) {$1$};
                    \node[rectangle, scale=0.7] (1c) at (2.4, 0.0) {$1$};
                    \node[rectangle, scale=0.7] (1d) at (3.6, 0.0) {$1$};
                    \node[rectangle, scale=0.7] (0a) at (4.4, 0.0) {$0$};
                    \node[rectangle, scale=0.7] (0b) at (5.6, 0.0) {$0$};
                    \node[rectangle, scale=0.7] (0c) at (6.4, 0.0) {$0$};
                    \node[rectangle, scale=0.7] (1e) at (7.6, 0.0) {$1$};

                    \draw[dashed, thick] (x) to (y1);
                    \draw[thick] (x) to (y2);
                    \draw[dashed, thick] (y1) to (z1);
                    \draw[thick] (y1) to (z2);
                    \draw[dashed, thick] (y2) to (z3);
                    \draw[thick] (y2) to (z4);
                    \draw[dashed, thick] (z1) to (1a);
                    \draw[thick] (z1) to (1b);
                    \draw[dashed, thick] (z2) to (1c);
                    \draw[thick] (z2) to (1d);
                    \draw[dashed, thick] (z3) to (0a);
                    \draw[thick] (z3) to (0b);
                    \draw[dashed, thick] (z4) to (0c);
                    \draw[thick] (z4) to (1e);
                \end{tikzpicture}

            \end{figure}

        \column{0.4\textwidth}
            \begin{center}
                \small
                \begin{table}
                    \begin{tabular}{ccc|c}
                        $x$ & $y$ & $z$ & $f(x,y,z)$\\
                        \hline
                        $0$ & $0$ & $0$ & $1$\\
                        $0$ & $0$ & $1$ & $1$\\
                        $0$ & $1$ & $0$ & $1$\\
                        $0$ & $1$ & $1$ & $1$\\
                        $1$ & $0$ & $0$ & $0$\\
                        $1$ & $0$ & $1$ & $0$\\
                        $1$ & $1$ & $0$ & $0$\\
                        $1$ & $1$ & $1$ & $1$\\
                    \end{tabular}
                \end{table}
            \end{center}
    
    \end{columns}
    
\end{frame}
%-------------------------------------------

%-------------------------------------------
\begin{frame}{Semelhanças com tabelas-verdade}

    \begin{outline}
        \1 Árvores de Decisão Binárias são semelhantes às tabelas-verdade em relação ao tamanho
            \2[-] se $f$ depender de $n$ variáveis booleanas, a árvore correspondente terá pelo menos $2^{n+1}-1$ nós (contra as $2^n$ linhas da tabela verdade)
        
        \vspace{1em}
        
        \1 Mas muitas vezes elas contêm redundâncias que podem ser exploradas
    \end{outline}      
   
\end{frame}
%-------------------------------------------

%-------------------------------------------
\begin{frame}{Primeira simplificação}

    \begin{figure}

        \begin{tikzpicture}
            [scale=0.9,auto=left,
                every node/.style={circle, minimum height=4mm, draw=black, thick, align=center, text depth = 0pt, transform shape},
                every edge/.append style={draw=blue, thick}
            ]
            %\draw[help lines] (0,0) grid (12,6);
          
            % Leftmost graph
            \node (x) at (3.0, 5.0)  {$x$};
            \node (y1) at (1.5, 3.0)  {$y$};
            \node (y2) at (4.5, 3.0)  {$y$};
            \node[rectangle, scale=0.7] (1) at (0.5, 1.0) {$1$};
            \node[rectangle, scale=0.7] (0a) at (2.5, 1.0) {$0$};
            \node[rectangle, scale=0.7] (0b) at (3.5, 1.0) {$0$};
            \node[rectangle, scale=0.7] (0c) at (5.5, 1.0) {$0$};

            \draw[dashed, thick] (x) to (y1);
            \draw[thick] (x) to (y2);
            \draw[dashed, thick] (y1) to (1);
            \draw[thick] (y1) to (0a);
            \draw[dashed, thick] (y2) to (0b);
            \draw[thick] (y2) to (0c);

            \uncover<2->{
                \node[rectangle, scale=0.7, blue] (0a) at (2.5, 1.0) {$0$};
                \node[rectangle, scale=0.7, red] (0b) at (3.5, 1.0) {$0$};
                \node[rectangle, scale=0.7, red] (0c) at (5.5, 1.0) {$0$};

                \draw[dashed, thick, blue] (y2) to (0b);
                \draw[thick, blue] (y2) to (0c);
            }
            
            \uncover<3->{
                % Arrow with text
                \node[fill=white, single arrow, draw=black] at (6.5,4.0) {\scriptsize Simplificação C1};
                
                % Rightmost graph
                \node (x_) at (10.0, 5.0)  {$x$};
                \node (y1_) at (8.5, 3.0)  {$y$};
                \node (y2_) at (11.5, 3.0)  {$y$};
                \node[rectangle, scale=0.7] (1_) at (7.5, 1.0) {$1$};
                \node[rectangle, scale=0.7, blue] (0_) at (9.5, 1.0) {$0$};

                \draw[dashed, thick] (x_) to (y1_);
                \draw[thick] (x_) to (y2_);
                \draw[dashed, thick] (y1_) to (1_);
                \draw[thick] (y1_) to (0_);
                \draw[bend left, thick, blue] (y2_) to (0_);
                \draw[bend right, dashed, thick, blue] (y2_) to (0_);
            }
        \end{tikzpicture}

    \end{figure}

    \begin{quotation}
    \noindent\footnotesize\justifying
        \textbf{C1. Remoção de terminais duplicados}\\
        Se há mais de um nó terminal $0$, todas as arestas que apontam para tais nós são redirecionadas para apenas um deles. O processo é então repetido para os nós terminais $1$
    \end{quotation}    
    
\end{frame}
%-------------------------------------------

%-------------------------------------------
\begin{frame}{Segunda simplificação}

    \begin{figure}

        \begin{tikzpicture}
            [scale=0.9,auto=left,
                every node/.style={circle, minimum height=4mm, draw=black, thick, align=center, text depth = 0pt, transform shape},
                every edge/.append style={draw=blue, thick}
            ]
            %\draw[help lines] (0,0) grid (12,6);
          
            % Leftmost graph
            \node (x) at (3.0, 5.0)  {$x$};
            \node (y1) at (1.5, 3.0)  {$y$};
            \node (y2) at (4.5, 3.0)  {$y$};
            \node[rectangle, scale=0.7] (1) at (0.5, 1.0) {$1$};
            \node[rectangle, scale=0.7] (0) at (2.5, 1.0) {$0$};

            \draw[dashed, thick] (x) to (y1);
            \draw[thick] (x) to (y2);
            \draw[dashed, thick] (y1) to (1);
            \draw[thick] (y1) to (0);
            \draw[bend left, thick] (y2) to (0);
            \draw[bend right, dashed, thick] (y2) to (0);
            
            \uncover<2->{
                \node[red] (y2) at (4.5, 3.0)  {$y$};
                \node[rectangle, scale=0.7, blue] (0) at (2.5, 1.0) {$0$};
                
                \draw[thick, blue] (x) to (y2);
                \draw[bend left, thick, red] (y2) to (0);
                \draw[bend right, dashed, thick, red] (y2) to (0);
            }
            
            \uncover<3->{
                % Arrow with text
                \node[fill=white, single arrow, draw=black] at (6.5,4.0) {\scriptsize Simplificação C2};
                
                % Rightmost graph
                \node (x_) at (10.0, 5.0)  {$x$};
                \node (y1_) at (8.5, 3.0)  {$y$};
                \node[rectangle, scale=0.7] (1_) at (7.5, 1.0) {$1$};
                \node[rectangle, scale=0.7, blue] (0_) at (9.5, 1.0) {$0$};

                \draw[dashed, thick] (x_) to (y1_);
                \draw[thick, blue] (x_) to (0_);
                \draw[dashed, thick] (y1_) to (1_);
                \draw[thick] (y1_) to (0_);
            }
        \end{tikzpicture}

    \end{figure}
    
    \begin{quotation}
    \noindent\footnotesize\justifying
        \textbf{C2. Remoção de testes redundantes}\\
        Se ambas as arestas de um nó $n$ apontam para o mesmo nó $m$, o nó $n$ é eliminado e todas as arestas que nele chegavam são redirecionadas diretamente para o nó $m$.
    \end{quotation}
    
\end{frame}
%-------------------------------------------

%-------------------------------------------
\begin{frame}{Terceira simplificação}

    \begin{figure}

        \begin{tikzpicture}
            [scale=0.8,auto=left,
                every node/.style={circle, minimum height=4mm, draw=black, thick, align=center, text depth = 0pt, transform shape},
                every edge/.append style={draw=blue, thick}
            ]
            %\draw[help lines] (0,0) grid (12,6);
          
            % Leftmost graph
            \node (z) at (3.0, 5.5)  {$z$};
            \node (x1) at (1.5, 4.0)  {$x$};
            \node (x2) at (4.5, 4.0)  {$x$};
            
            \node (y1) at (0.6, 2.5)  {$y$};
            \node (y2) at (2.4, 2.5)  {$y$};
            \node (y3) at (3.6, 2.5)  {$y$};
            \node (y4) at (5.4, 2.5)  {$y$};
            
            \node[rectangle, scale=0.7] (0) at (1.5, 0.5) {$0$};
            \node[rectangle, scale=0.7] (1) at (4.5, 0.5) {$1$};

            \draw[dashed, thick] (z) to (x1);
            \draw[thick] (z) to (x2);
            \draw[dashed, thick] (x1) to (y1);
            \draw[thick] (x1) to (y2);
            \draw[dashed, thick] (x2) to (y3);
            \draw[thick] (x2) to (y4);
            
            \draw[dashed, thick] (y1) to (0);
            \draw[thick] (y1) to (1);
            \draw[dashed, thick] (y2) to (0);
            \draw[thick] (y2) to (1);
            \draw[dashed, thick] (y3) to (0);
            \draw[thick] (y3) to (1);
            \draw[thick] (y4) to (0);
            \draw[dashed, thick] (y4) to (1);
            
            \uncover<2->{
                \node[blue] (y2) at (2.4, 2.5)  {$y$};
                \node[red] (y3) at (3.6, 2.5)  {$y$};
                
                \draw[dashed, thick, blue] (x2) to (y3);
                \draw[dashed, thick, blue] (y2) to (0);
                \draw[thick, blue] (y2) to (1);                
                \draw[dashed, thick, red] (y3) to (0);
                \draw[thick, red] (y3) to (1);
            }
            
            \uncover<3->{
                % Arrow with text
                \node[fill=white, single arrow, draw=black] at (6.7,4.0) {\scriptsize Simplificação C3};
                
                % Rightmost graph
                \node (z_) at (10.5, 5.5)  {$z$};
                \node (x1_) at (9.0, 4.0)  {$x$};
                \node (x2_) at (12.0, 4.0)  {$x$};
                
                \node (y1_) at (8.1, 2.5)  {$y$};
                \node[blue] (y2_) at (9.9, 2.5)  {$y$};
                \node (y4_) at (12.9, 2.5)  {$y$};
                
                \node[rectangle, scale=0.7] (0_) at (9.0, 0.5) {$0$};
                \node[rectangle, scale=0.7] (1_) at (12.0, 0.5) {$1$};

                \draw[dashed, thick] (z_) to (x1_);
                \draw[thick] (z_) to (x2_);
                \draw[dashed, thick] (x1_) to (y1_);
                \draw[thick] (x1_) to (y2_);
                \draw[dashed, thick, blue] (x2_) to (y2_);
                \draw[thick] (x2_) to (y4_);
                
                \draw[dashed, thick] (y1_) to (0_);
                \draw[thick] (y1_) to (1_);
                \draw[dashed, thick, blue] (y2_) to (0_);
                \draw[thick, blue] (y2_) to (1_);
                \draw[thick] (y4_) to (0_);
                \draw[dashed, thick] (y4_) to (1_);
            }
        \end{tikzpicture}

    \end{figure}
    
    \begin{quotation}
    \noindent\footnotesize\justifying
        \textbf{C3. Remoção de nós não-terminais duplicados}\\
        Se dois nós distintos $n$ e $m$ são raízes de subárvores idênticas, um dos nós é eliminado e todas as arestas que nele chegavam são redirecionadas para o outro
    \end{quotation}
    
\end{frame}
%-------------------------------------------

%-------------------------------------------
\begin{frame}{Processo de redução}

    \begin{figure}

        \begin{tikzpicture}
            [scale=0.8,auto=left,
                every node/.style={circle, minimum height=4mm, draw=black, thick, align=center, text depth = 0pt, transform shape},
                every edge/.append style={draw=blue, thick}
            ]
            %\draw[help lines] (0,0) grid (12,6);
          
            % Leftmost graph
            \node (z) at (3.0, 5.5)  {$z$};
            \node (x1) at (1.5, 4.0)  {$x$};
            \node (x2) at (4.5, 4.0)  {$x$};
            
            \node (y1) at (0.6, 2.5)  {$y$};
            \node (y2) at (2.4, 2.5)  {$y$};
            \node (y4) at (5.4, 2.5)  {$y$};
            
            \node[rectangle, scale=0.7] (0) at (1.5, 0.5) {$0$};
            \node[rectangle, scale=0.7] (1) at (4.5, 0.5) {$1$};

            \draw[dashed, thick] (z) to (x1);
            \draw[thick] (z) to (x2);
            \draw[dashed, thick] (x1) to (y1);
            \draw[thick] (x1) to (y2);
            \draw[dashed, thick] (x2) to (y2);
            \draw[thick] (x2) to (y4);
            
            \draw[dashed, thick] (y1) to (0);
            \draw[thick] (y1) to (1);
            \draw[dashed, thick] (y2) to (0);
            \draw[thick] (y2) to (1);
            \draw[thick] (y4) to (0);
            \draw[dashed, thick] (y4) to (1);
            
            \uncover<2->{
                \node[red] (y1) at (0.6, 2.5)  {$y$};
                \node[blue] (y2) at (2.4, 2.5)  {$y$};
                
                \draw[dashed, thick, blue] (x1) to (y1);
                \draw[dashed, thick, red] (y1) to (0);
                \draw[thick, red] (y1) to (1);
            }
            
            \uncover<3->{
                \node[red] (x1) at (1.5, 4.0)  {$x$};
                \draw[dashed, thick, blue] (z) to (x1);
                \draw[dashed, thick, red] (x1) to (y1);
                \draw[thick, red] (x1) to (y2);
            }

            \uncover<4->{
                % Arrow with text
                \node[fill=white, single arrow, draw=black] at (6.7,4.0) {\small C3 + C2};
                
                % Rightmost graph
                \node (z_) at (9.0, 5.5)  {$z$};
                \node (x2_) at (10.5, 4.0)  {$x$};
                
                \node[blue] (y2_) at (8.4, 2.5)  {$y$};
                \node (y4_) at (11.4, 2.5)  {$y$};
                
                \node[rectangle, scale=0.7] (0_) at (7.5, 0.5) {$0$};
                \node[rectangle, scale=0.7] (1_) at (10.5, 0.5) {$1$};

                \draw[dashed, thick, blue] (z_) to (y2_);
                \draw[thick] (z_) to (x2_);
                \draw[dashed, thick] (x2_) to (y2_);
                \draw[thick] (x2_) to (y4_);
                
                \draw[dashed, thick] (y2_) to (0_);
                \draw[thick] (y2_) to (1_);
                \draw[thick] (y4_) to (0_);
                \draw[dashed, thick] (y4_) to (1_);
            }
        \end{tikzpicture}

    \end{figure}
    
    \begin{quotation}
    \noindent\footnotesize\justifying
        As simplificações são encadeadas até não mais ser possível. O exemplo anterior é completamente reduzido após a eliminação de um dos nós $y$ duplicados (C3) seguida da eliminação de um ponto de decisão $x$ redundante (C2)
    \end{quotation}
    
\end{frame}
%-------------------------------------------

%-------------------------------------------
\begin{frame}{Exercício 1}

    {\small
    Reduza a árvore de decisão binária da função $f(x,y,z) \defeq \overline{x} + (y \cdot z)$ apresentada anteriormente:}
   
    \begin{columns}[c]
        \column{0.5\textwidth}

            \begin{figure}

                \begin{tikzpicture}
                    [scale=0.7,auto=left,
                        every node/.style={circle, minimum height=4mm, draw=black, thick, align=center, text depth = 0pt, transform shape},
                        every edge/.append style={draw=blue, thick}
                    ]
                    %\draw[help lines] (0,0) grid (8,6);
                  
                    \node (p) at (4.0, 6.0)  {$x$};
                    \node (q1) at (2.0, 4.0)  {$y$};
                    \node (q2) at (6.0, 4.0)  {$y$};
                    \node (r1) at (1.0, 2.0)  {$z$};
                    \node (r2) at (3.0, 2.0)  {$z$};
                    \node (r3) at (5.0, 2.0)  {$z$};
                    \node (r4) at (7.0, 2.0)  {$z$};
                    
                    \node[rectangle, scale=0.7] (1a) at (0.4, 0.0) {$1$};
                    \node[rectangle, scale=0.7] (1b) at (1.6, 0.0) {$1$};
                    \node[rectangle, scale=0.7] (1c) at (2.4, 0.0) {$1$};
                    \node[rectangle, scale=0.7] (1d) at (3.6, 0.0) {$1$};
                    \node[rectangle, scale=0.7] (0a) at (4.4, 0.0) {$0$};
                    \node[rectangle, scale=0.7] (0b) at (5.6, 0.0) {$0$};
                    \node[rectangle, scale=0.7] (0c) at (6.4, 0.0) {$0$};
                    \node[rectangle, scale=0.7] (1e) at (7.6, 0.0) {$1$};

                    \draw[dashed, thick] (p) to (q1);
                    \draw[thick] (p) to (q2);
                    \draw[dashed, thick] (q1) to (r1);
                    \draw[thick] (q1) to (r2);
                    \draw[dashed, thick] (q2) to (r3);
                    \draw[thick] (q2) to (r4);
                    \draw[dashed, thick] (r1) to (1a);
                    \draw[thick] (r1) to (1b);
                    \draw[dashed, thick] (r2) to (1c);
                    \draw[thick] (r2) to (1d);
                    \draw[dashed, thick] (r3) to (0a);
                    \draw[thick] (r3) to (0b);
                    \draw[dashed, thick] (r4) to (0c);
                    \draw[thick] (r4) to (1e);
                \end{tikzpicture}

            \end{figure}

        \column{0.5\textwidth}
            
            {\small
                Resumo das simplificações:
                {\scriptsize
                    \begin{outline}
                        \1[\textbf{C1}.] Remoção de nós terminais duplicados
                        \1[\textbf{C2}.] Remoção de testes redundantes
                        \1[\textbf{C3}.] Remoção de nós não-terminais duplicados
                    \end{outline}
                }
            }
    
    \end{columns}
    
\end{frame}
%-------------------------------------------

%-------------------------------------------
\begin{frame}{Solução -- 1º passo}

    \begin{figure}

        \begin{tikzpicture}
            [scale=0.7,auto=left,
                every node/.style={circle, minimum height=4mm, draw=black, thick, align=center, text depth = 0pt, transform shape},
                every edge/.append style={draw=blue, thick}
            ]
            %\draw[help lines] (0,0) grid (16,6);
          
            %Leftmost graph
            \node (p) at (4.0, 6.0)  {$x$};
            \node (q1) at (2.0, 4.0)  {$y$};
            \node (q2) at (6.0, 4.0)  {$y$};
            \node (r1) at (1.0, 2.0)  {$z$};
            \node (r2) at (3.0, 2.0)  {$z$};
            \node (r3) at (5.0, 2.0)  {$z$};
            \node (r4) at (7.0, 2.0)  {$z$};
            
            \node[rectangle, scale=0.7, blue] (1a) at (0.4, 0.0) {$1$};
            \node[rectangle, scale=0.7, red] (1b) at (1.6, 0.0) {$1$};
            \node[rectangle, scale=0.7, red] (1c) at (2.4, 0.0) {$1$};
            \node[rectangle, scale=0.7, red] (1d) at (3.6, 0.0) {$1$};
            \node[rectangle, scale=0.7, red] (0a) at (4.4, 0.0) {$0$};
            \node[rectangle, scale=0.7, red] (0b) at (5.6, 0.0) {$0$};
            \node[rectangle, scale=0.7, blue] (0c) at (6.4, 0.0) {$0$};
            \node[rectangle, scale=0.7, red] (1e) at (7.6, 0.0) {$1$};

            \draw[dashed, thick] (p) to (q1);
            \draw[thick] (p) to (q2);
            \draw[dashed, thick] (q1) to (r1);
            \draw[thick] (q1) to (r2);
            \draw[dashed, thick] (q2) to (r3);
            \draw[thick] (q2) to (r4);
            \draw[dashed, thick] (r1) to (1a);
            \draw[thick] (r1) to (1b);
            \draw[dashed, thick] (r2) to (1c);
            \draw[thick] (r2) to (1d);
            \draw[dashed, thick] (r3) to (0a);
            \draw[thick] (r3) to (0b);
            \draw[dashed, thick] (r4) to (0c);
            \draw[thick] (r4) to (1e);
            
            % Arrow with text
            \node[fill=white, single arrow, draw=black] at (8.5,4.0) {\small C1};
                
            % Rightmost graph
            \node (p_) at (13.0, 6.0)  {$x$};
            \node (q1_) at (11.0, 4.0) {$y$};
            \node (q2_) at (15.0, 4.0) {$y$};
            \node (r1_) at (10.0, 2.0) {$z$};
            \node (r2_) at (12.0, 2.0) {$z$};
            \node (r3_) at (14.0, 2.0) {$z$};
            \node (r4_) at (16.0, 2.0) {$z$};
            
            \node[rectangle, scale=0.7, blue] (1_) at (11.0, 0.0) {$1$};
            \node[rectangle, scale=0.7, blue] (0_) at (15.0, 0.0) {$0$};

            \draw[dashed, thick] (p_) to (q1_);
            \draw[thick] (p_) to (q2_);
            \draw[dashed, thick] (q1_) to (r1_);
            \draw[thick] (q1_) to (r2_);
            \draw[dashed, thick] (q2_) to (r3_);
            \draw[thick] (q2_) to (r4_);
            \draw[bend right=40, dashed, thick] (r1_) to (1_);
            \draw[bend left=20, thick] (r1_) to (1_);
            \draw[bend right=20, dashed, thick] (r2_) to (1_);
            \draw[bend left=40, thick] (r2_) to (1_);
            \draw[bend right=40, dashed, thick] (r3_) to (0_);
            \draw[bend left=20, thick] (r3_) to (0_);
            \draw[bend right=20, dashed, thick] (r4_) to (0_);
            \draw[bend left=90, thick] (r4_) to (1_);
                      
        \end{tikzpicture}

    \end{figure}
    
\end{frame}
%-------------------------------------------

%-------------------------------------------
\begin{frame}{Solução -- 2º passo}

    \begin{figure}

        \begin{tikzpicture}
            [scale=0.7,auto=left,
                every node/.style={circle, minimum height=4mm, draw=black, thick, align=center, text depth = 0pt, transform shape},
                every edge/.append style={draw=blue, thick}
            ]
            %\draw[help lines] (0,0) grid (16,6);
          
            %Leftmost graph
            \node (p) at (4.0, 6.0)       {$x$};
            \node (q1) at (2.0, 4.0)      {$y$};
            \node (q2) at (6.0, 4.0)      {$y$};
            \node[red] (r1) at (1.0, 2.0) {$z$};
            \node[red] (r2) at (3.0, 2.0) {$z$};
            \node[red] (r3) at (5.0, 2.0) {$z$};
            \node (r4) at (7.0, 2.0)      {$z$};
            
            \node[rectangle, scale=0.7] (1) at (2.0, 0.0) {$1$};
            \node[rectangle, scale=0.7] (0) at (6.0, 0.0) {$0$};

            \draw[dashed, thick] (p) to (q1);
            \draw[thick] (p) to (q2);
            \draw[dashed, thick, blue] (q1) to (r1);
            \draw[thick, blue] (q1) to (r2);
            \draw[dashed, thick, blue] (q2) to (r3);
            \draw[thick] (q2) to (r4);
            \draw[bend right=40, dashed, thick, red] (r1) to (1);
            \draw[bend left=20, thick, red] (r1) to (1);
            \draw[bend right=20, dashed, thick, red] (r2) to (1);
            \draw[bend left=40, thick, red] (r2) to (1);
            \draw[bend right=40, dashed, thick, red] (r3) to (0);
            \draw[bend left=20, thick, red] (r3) to (0);
            \draw[bend right=20, dashed, thick] (r4) to (0);
            \draw[bend left=90, thick] (r4) to (1);
            
            % Arrow with text
            \node[fill=white, single arrow, draw=black] at (8.5,4.0) {\small C2};
                
            % Rightmost graph
            \node (p_) at (13.0, 6.0)  {$x$};
            \node (q1_) at (11.0, 4.0) {$y$};
            \node (q2_) at (15.0, 4.0) {$y$};
            \node (r4_) at (16.0, 2.0) {$z$};
            
            \node[rectangle, scale=0.7] (1_) at (11.0, 0.0) {$1$};
            \node[rectangle, scale=0.7] (0_) at (15.0, 0.0) {$0$};

            \draw[dashed, thick] (p_) to (q1_);
            \draw[thick] (p_) to (q2_);
            \draw[bend right, dashed, thick, blue] (q1_) to (1_);
            \draw[bend left, thick, blue] (q1_) to (1_);
            \draw[bend right, dashed, thick, blue] (q2_) to (0_);
            \draw[thick] (q2_) to (r4_);
            \draw[bend right=20, dashed, thick] (r4_) to (0_);
            \draw[bend left=90, thick] (r4_) to (1_);
                      
        \end{tikzpicture}

    \end{figure}
    
\end{frame}
%-------------------------------------------

%-------------------------------------------
\begin{frame}{Solução -- 3º passo}

    \begin{figure}

        \begin{tikzpicture}
            [scale=0.7,auto=left,
                every node/.style={circle, minimum height=4mm, draw=black, thick, align=center, text depth = 0pt, transform shape},
                every edge/.append style={draw=blue, thick}
            ]
            %\draw[help lines] (0,0) grid (16,6);
          
            %Leftmost graph
            \node (p_) at (4.0, 6.0)       {$x$};
            \node[red] (q1_) at (2.0, 4.0) {$y$};
            \node (q2_) at (6.0, 4.0)      {$y$};
            \node (r4_) at (7.0, 2.0)      {$z$};
            
            \node[rectangle, scale=0.7] (1_) at (2.0, 0.0) {$1$};
            \node[rectangle, scale=0.7] (0_) at (6.0, 0.0) {$0$};

            \draw[dashed, thick, blue] (p_) to (q1_);
            \draw[thick] (p_) to (q2_);
            \draw[bend right, dashed, thick, red] (q1_) to (1_);
            \draw[bend left, thick, red] (q1_) to (1_);
            \draw[bend right, dashed, thick] (q2_) to (0_);
            \draw[thick] (q2_) to (r4_);
            \draw[bend right=20, dashed, thick] (r4_) to (0_);
            \draw[bend left=90, thick] (r4_) to (1_);
            
            % Arrow with text
            \node[fill=white, single arrow, draw=black] at (8.5,4.0) {\small C2};
                
            % Rightmost graph
            \node (p_) at (13.0, 6.0)  {$x$};
            \node (q2_) at (15.0, 4.0) {$y$};
            \node (r4_) at (16.0, 2.0) {$z$};
            
            \node[rectangle, scale=0.7] (1_) at (11.0, 0.0) {$1$};
            \node[rectangle, scale=0.7] (0_) at (15.0, 0.0) {$0$};

            \draw[bend right, dashed, thick, blue] (p_) to (1_);
            \draw[thick] (p_) to (q2_);
            \draw[bend right, dashed, thick] (q2_) to (0_);
            \draw[thick] (q2_) to (r4_);
            \draw[bend right=20, dashed, thick] (r4_) to (0_);
            \draw[bend left=90, thick] (r4_) to (1_);
        \end{tikzpicture}

    \end{figure}
    
\end{frame}
%-------------------------------------------

%-------------------------------------------
\begin{frame}{Comparando com a tabela-verdade}

    Função booleana: $f(x,y,z) \defeq \overline{x} + (y \cdot z)$:
  
    \begin{columns}[c]
        \column{0.6\textwidth}

            \begin{figure}

                \begin{tikzpicture}
                    [scale=0.9,auto=left,
                        every node/.style={circle, minimum height=4mm, draw=black, thick, align=center, text depth = 0pt, transform shape},
                        every edge/.append style={draw=blue, thick}
                    ]
                    %\draw[help lines] (0,0) grid (8,6);
                  
                    \node[magenta] (p) at (3.0, 5.0) {$x$};
                    \node[Maroon] (q) at (6.0, 4.0)  {$y$};
                    \node[Maroon] (r) at (4.0, 2.0)  {$z$};
                    
                    \node[rectangle, scale=0.7] (1) at (2.0, 1.0) {$1$};
                    \node[rectangle, scale=0.7] (0) at (6.0, 1.0) {$0$};

                    \draw[dashed, thick, magenta] (p) to (1);
                    \draw[thick] (p) to (q);
                    \draw[dashed, thick] (q) to (0);
                    \draw[thick, Maroon] (q) to (r);
                    \draw[dashed, thick] (r) to (0);
                    \draw[thick, Maroon] (r) to (1);
                \end{tikzpicture}

            \end{figure}

        \column{0.4\textwidth}
            \begin{center}
                \small
                \begin{table}
                    \begin{tabular}{ccc|c}
                        \textcolor{magenta}{$x$} & $y$ & $z$ & $f(x,y,z)$\\
                        \hline
                        \textcolor{magenta}{$0$} & $0$ & $0$ & \textcolor{magenta}{$1$}\\
                        \textcolor{magenta}{$0$} & $0$ & $1$ & \textcolor{magenta}{$1$}\\
                        \textcolor{magenta}{$0$} & $1$ & $0$ & \textcolor{magenta}{$1$}\\
                        \textcolor{magenta}{$0$} & $1$ & $1$ & \textcolor{magenta}{$1$}\\
                        $1$ & $0$ & $0$ & $0$\\
                        $1$ & $0$ & $1$ & $0$\\
                        $1$ & $1$ & $0$ & $0$\\
                        $1$ & \textcolor{Maroon}{$1$} & \textcolor{Maroon}{$1$} & \textcolor{Maroon}{$1$}\\
                    \end{tabular}
                \end{table}
            \end{center}
    
    \end{columns}
    
\end{frame}
%-------------------------------------------

%-------------------------------------------
\begin{frame}{\uppercase{BDD}s}

    \begin{center}
        \shadowbox{
            \begin{minipage}[h]{0.9\linewidth}
                \noindent
                A redução faz com que as árvores se tornem grafos. Por isso, passam a ser chamados de \textbf{Diagramas de Decisão Binários (BDDs)}.
            \end{minipage}
        }
    \end{center}

\end{frame}
%-------------------------------------------

%-------------------------------------------
\begin{frame}{Definição: gda}

    \begin{block}{\textbf{Definição 6.4}}
        Um grafo direcionado é um conjunto $\mathit{G}$ e uma relação binária $\rightarrow$ em $\mathit{G}:~\rightarrow \subseteq \mathit{G} \times \mathit{G}$. Um ciclo em um grafo direcionado é um caminho finito no grafo que começa e termina no mesmo nó, isto é, um caminho da forma $v_1 \rightarrow v_2 \rightarrow ... \rightarrow v_n \rightarrow v_1$. Um grafo direcionado acíclico (gda) é um grafo direcionado que não contém nenhum ciclo. Um nó em um gda é dito inicial se não há arestas apontando para ele. Um nó é dito terminal se não há arestas saindo dele.
    \end{block}

\end{frame}
%-------------------------------------------

%-------------------------------------------
\begin{frame}{Definição: \uppercase{BDD}s}

    \begin{block}{\textbf{Definição 6.5}}
        Um diagrama de decisão binário (BDD) é um gda finito com um único nó inicial, onde todos os nós terminais são marcados com $0$ ou $1$ e todos os nós não-terminais são marcados com uma variável booleana. Cada nó não-terminal tem exatamente duas arestas saindo dele, uma marcada com $0$ e outra com $1$ (representadas como uma linha pontilhada e uma linha sólida, respectivamente).
    \end{block}

\end{frame}
%-------------------------------------------

%-------------------------------------------
\begin{frame}{\uppercase{BDD} como gda}

    \begin{outline}
        \1 Por convenção, as linhas sólidas ou pontilhadas de um BDD são sempre consideradas como indo para baixo
            \2[-] por isso eles são grafos direcionados
        
        \vspace{1em}
        
        \1 Os BDDs são acíclicos (gda) e têm um único nó inicial

        \vspace{1em}
        
        \1 As simplificações C1--C3 preservam essas propriedades
            \2[-] BDDs totalmente reduzidos têm 1 ou 2 nós terminais
    \end{outline}

\end{frame}
%-------------------------------------------

%-------------------------------------------
\begin{frame}{\uppercase{BDD}s elementares}

    \begin{figure}

        \caption{O BDD $B_0$ representa a função booleana constante $0$; analogamente, o BDD $B_1$ representa a função booleana constante $1$; e, finalmente, o BDD $B_x$ representa a variável booleana $x$}
    
        \begin{tikzpicture}
            [scale=1.0,auto=left,
                every node/.style={circle, minimum height=4mm, draw=black, thick, align=center, text depth = 0pt, transform shape},
                every edge/.append style={draw=blue, thick}
            ]
            %\draw[help lines] (0,0) grid (8,6);

            \node[draw=none, scale=0.6] at (0.0, 2.8) {$B_0$};
            \node[rectangle, scale=0.7] at (0.0, 2.0) {$0$};

            \node[draw=none, scale=0.6] at (3.0, 2.8) {$B_1$};
            \node[rectangle, scale=0.7] at (3.0, 2.0) {$1$};
            
            \node[draw=none, scale=0.6] at (6.0, 2.8) {$B_x$};
            \node (x) at (6.0, 2.0) {$x$};
            \node[rectangle, scale=0.7] (0) at (5.0, 0.0) {$0$};
            \node[rectangle, scale=0.7] (1) at (7.0, 0.0) {$1$};

            \draw[dashed, thick] (x) to (0);
            \draw[thick] (x) to (1);
        \end{tikzpicture}

    \end{figure}


\end{frame}
%-------------------------------------------

%-------------------------------------------
\begin{frame}{Verificações sobre \uppercase{BDD}s}

    \begin{outline}
        \1 \textbf{Satisfação}. Um BDD representa uma função que pode ser satisfeita se um nó terminal $1$ pode ser acessado da raiz por meio de um caminho consistente
        
        \vspace{1em}
        
        \1 \textbf{Validade}. Um BDD representa uma função válida se nenhum ponto terminal $0$ é acessível por um caminho consistente
    \end{outline}

\end{frame}
%-------------------------------------------

%-------------------------------------------
\begin{frame}{Exemplos óbvios}

    \begin{columns}[c]
        \column{0.3\textwidth}

            \begin{figure}

                \caption{$f(x,y) \defeq x \cdot y$}
            
                \begin{tikzpicture}
                    [scale=0.8,auto=left,
                        every node/.style={circle, minimum height=4mm, draw=black, thick, align=center, text depth = 0pt, transform shape},
                        every edge/.append style={draw=blue, thick}
                    ]
                    %\draw[help lines] (0,0) grid (6,6);
                  
                    \node (x) at (3.0, 5.0) {$x$};
                    \node (y) at (4.5, 3.0) {$y$};
                    \node[rectangle, scale=0.7] (0) at (1.5, 1.0) {$0$};
                    \node[rectangle, scale=0.7] (1) at (4.5, 1.0) {$1$};

                    \draw[dashed, thick] (x) to (0);
                    \draw[thick] (x) to (y);
                    \draw[dashed, thick] (y) to (0);
                    \draw[thick] (y) to (1);

                \end{tikzpicture}

            \end{figure}

        \column{0.3\textwidth}
            \begin{figure}

                \caption{$g(x) \defeq x + \overline{x}$}
            
                \begin{tikzpicture}
                    [scale=0.8,auto=left,
                        every node/.style={circle, minimum height=4mm, draw=black, thick, align=center, text depth = 0pt, transform shape},
                        every edge/.append style={draw=blue, thick}
                    ]
                    %\draw[help lines] (0,0) grid (6,6);

                    \node[magenta] (x) at (3.0, 5.0) {$x$};
                    \node[rectangle, scale=0.7] (1) at (3.0, 1.0) {$1$};

                    \draw[dashed, thick, bend right, magenta] (x) to (1);
                    \draw[thick, bend left, magenta] (x) to (1);
                    
                \end{tikzpicture}

            \end{figure}

        \column{0.3\textwidth}
            \begin{figure}

                \caption{$h(y) \defeq y \cdot \overline{y}$}
            
                \begin{tikzpicture}
                    [scale=0.8,auto=left,
                        every node/.style={circle, minimum height=4mm, draw=black, thick, align=center, text depth = 0pt, transform shape},
                        every edge/.append style={draw=blue, thick}
                    ]
                    %\draw[help lines] (0,0) grid (6,6);

                    \node[magenta] (y) at (3.0, 5.0) {$y$};
                    \node[rectangle, scale=0.7] (0) at (3.0, 1.0) {$0$};

                    \draw[dashed, thick, bend right, magenta] (y) to (0);
                    \draw[thick, bend left, magenta] (y) to (0);
                    
                \end{tikzpicture}

            \end{figure}
        
    \end{columns}

\end{frame}
%-------------------------------------------

%-------------------------------------------
\begin{frame}{Operações sobre \uppercase{BDD}s}

    \begin{outline}
        \small
        \1 \textbf{Operação de negação ($\bar{~}$)}. Obtem-se um BDD que representa $\overline{f}$ substituindo todos os terminais $0$ em $B_f$ por terminais $1$ e vice-versa
        
        \vspace{1em}
        
        \1 \textbf{Operação de conjunção ($\cdot$)}. Obtem-se um BDD que representa $f \cdot g$ substituindo todos os nós terminais $1$ em $B_f$ diretamente por uma cópia de $B_g$
        
        \vspace{1em}
        
        \1 \textbf{Operação de disjunção ($+$)}. Obtem-se um BDD que representa $f + g$ substituindo todos os nós terminais $0$ em $B_f$ diretamente por uma cópia de $B_g$
    \end{outline}

\end{frame}
%-------------------------------------------

%-------------------------------------------
\begin{frame}{Exemplo da negação}

    \begin{columns}[c]
        \column{0.5\textwidth}

            \begin{figure}

                \caption{$f(x,y) \defeq x \cdot y$}
            
                \begin{tikzpicture}
                    [scale=0.8,auto=left,
                        every node/.style={circle, minimum height=4mm, draw=black, thick, align=center, text depth = 0pt, transform shape},
                        every edge/.append style={draw=blue, thick}
                    ]
                    %\draw[help lines] (0,0) grid (6,6);
                  
                    \node (x) at (3.0, 5.0) {$x$};
                    \node (y) at (4.5, 3.0) {$y$};
                    \node[rectangle, scale=0.7, magenta] (0) at (1.5, 1.0) {$0$};
                    \node[rectangle, scale=0.7, Maroon] (1) at (4.5, 1.0) {$1$};

                    \draw[dashed, thick] (x) to (0);
                    \draw[thick] (x) to (y);
                    \draw[dashed, thick] (y) to (0);
                    \draw[thick] (y) to (1);

                \end{tikzpicture}

            \end{figure}

        \column{0.5\textwidth}
            \begin{figure}

                \caption{$g(x,y) \defeq \overline{x \cdot y}$}
            
                \begin{tikzpicture}
                    [scale=0.8,auto=left,
                        every node/.style={circle, minimum height=4mm, draw=black, thick, align=center, text depth = 0pt, transform shape},
                        every edge/.append style={draw=blue, thick}
                    ]
                    %\draw[help lines] (0,0) grid (6,6);
                  
                    \node (x) at (3.0, 5.0) {$x$};
                    \node (y) at (4.5, 3.0) {$y$};
                    \node[rectangle, scale=0.7, Maroon] (1) at (1.5, 1.0) {$1$};
                    \node[rectangle, scale=0.7, magenta] (0) at (4.5, 1.0) {$0$};

                    \draw[dashed, thick] (x) to (1);
                    \draw[thick] (x) to (y);
                    \draw[dashed, thick] (y) to (1);
                    \draw[thick] (y) to (0);

                \end{tikzpicture}

            \end{figure}
        
    \end{columns}
    
\end{frame}
%-------------------------------------------

%-------------------------------------------
\begin{frame}{Exemplo da conjunção}

    \vspace{-1em}

    \begin{columns}[c]
        \column{0.5\textwidth}
        
            \begin{figure}

                \caption{$f(x,y) \defeq x \cdot y$}
            
                \begin{tikzpicture}
                    [scale=0.5,auto=left,
                        every node/.style={circle, minimum height=4mm, draw=black, thick, align=center, text depth = 0pt, transform shape},
                        every edge/.append style={draw=blue, thick}
                    ]
                    %\draw[help lines] (0,0) grid (6,6);
                    
                    \node (x) at (3.0, 5.0) {$x$};
                    \node (y) at (4.5, 3.0) {$y$};
                    \node[rectangle, scale=0.7] (0) at (1.5, 1.0) {$0$};
                    \node[rectangle, scale=0.7, red] (1) at (4.5, 1.0) {$1$};

                    \draw[dashed, thick] (x) to (0);
                    \draw[thick] (x) to (y);
                    \draw[dashed, thick] (y) to (0);
                    \draw[thick] (y) to (1);
                  
                \end{tikzpicture}

            \end{figure}

            \vspace{-2em}
            
            \begin{figure}

                \caption{$g(x,y) \defeq \overline{x} + y$}
            
                \begin{tikzpicture}
                    [scale=0.5,auto=left,
                        every node/.style={circle, minimum height=4mm, draw=black, thick, align=center, text depth = 0pt, transform shape},
                        every edge/.append style={draw=blue, thick}
                    ]
                    %\draw[help lines] (0,0) grid (6,6);
                    
                    \node[blue] (x) at (3.0, 5.0) {$x$};
                    \node (y) at (4.5, 3.0) {$y$};
                    \node[rectangle, scale=0.7] (1) at (1.5, 1.0) {$1$};
                    \node[rectangle, scale=0.7] (0) at (4.5, 1.0) {$0$};

                    \draw[dashed, thick] (x) to (1);
                    \draw[thick] (x) to (y);
                    \draw[dashed, thick] (y) to (0);
                    \draw[thick] (y) to (1);
                \end{tikzpicture}

            \end{figure}
            
        \column{0.5\textwidth}            

            \begin{figure}

                \caption{$h(x,y) \defeq (x \cdot y) \cdot (\overline{x} + y)$}
            
                \begin{tikzpicture}
                    [scale=0.5,auto=left,
                        every node/.style={circle, minimum height=4mm, draw=black, thick, align=center, text depth = 0pt, transform shape},
                        every edge/.append style={draw=blue, thick}
                    ]
                    %\draw[help lines] (0,0) grid (6,6);
                    
                    \node (x1) at (3.0, 5.0) {$x$};
                    \node (y1) at (4.5, 3.0) {$y$};
                    \node[rectangle, scale=0.7] (0a) at (1.5, 1.0) {$0$};

                    \node[blue] (x2) at (4.5, 1.0) {$x$};
                    \node (y2) at (6.0, -1.0) {$y$};
                    \node[rectangle, scale=0.7] (1) at (3.0, -3.0) {$1$};
                    \node[rectangle, scale=0.7] (0b) at (6.0, -3.0) {$0$};
                    
                    \draw[dashed, thick] (x1) to (0a);
                    \draw[thick] (x1) to (y1);
                    \draw[dashed, thick] (y1) to (0a);
                    \draw[thick] (y1) to (x2);

                    \draw[dashed, thick] (x2) to (1);
                    \draw[thick] (x2) to (y2);
                    \draw[dashed, thick] (y2) to (0b);
                    \draw[thick] (y2) to (1);
                    
                \end{tikzpicture}

            \end{figure}
    \end{columns}
    
\end{frame}
%-------------------------------------------

%-------------------------------------------
\begin{frame}{Exemplo da disjunção}

    \vspace{-1em}

    \begin{columns}[c]
        \column{0.5\textwidth}
        
            \begin{figure}

                \caption{$f(x,y) \defeq x \cdot y$}
            
                \begin{tikzpicture}
                    [scale=0.5,auto=left,
                        every node/.style={circle, minimum height=4mm, draw=black, thick, align=center, text depth = 0pt, transform shape},
                        every edge/.append style={draw=blue, thick}
                    ]
                    %\draw[help lines] (0,0) grid (6,6);
                    
                    \node (x) at (3.0, 5.0) {$x$};
                    \node (y) at (4.5, 3.0) {$y$};
                    \node[rectangle, scale=0.7, red] (0) at (1.5, 1.0) {$0$};
                    \node[rectangle, scale=0.7] (1) at (4.5, 1.0) {$1$};

                    \draw[dashed, thick] (x) to (0);
                    \draw[thick] (x) to (y);
                    \draw[dashed, thick] (y) to (0);
                    \draw[thick] (y) to (1);
                  
                \end{tikzpicture}

            \end{figure}

            \vspace{-2em}
            
            \begin{figure}

                \caption{$g(x,y) \defeq \overline{x} + y$}
            
                \begin{tikzpicture}
                    [scale=0.5,auto=left,
                        every node/.style={circle, minimum height=4mm, draw=black, thick, align=center, text depth = 0pt, transform shape},
                        every edge/.append style={draw=blue, thick}
                    ]
                    %\draw[help lines] (0,0) grid (6,6);
                    
                    \node[blue] (x) at (3.0, 5.0) {$x$};
                    \node (y) at (4.5, 3.0) {$y$};
                    \node[rectangle, scale=0.7] (1) at (1.5, 1.0) {$1$};
                    \node[rectangle, scale=0.7] (0) at (4.5, 1.0) {$0$};

                    \draw[dashed, thick] (x) to (1);
                    \draw[thick] (x) to (y);
                    \draw[dashed, thick] (y) to (0);
                    \draw[thick] (y) to (1);
                \end{tikzpicture}

            \end{figure}
            
        \column{0.5\textwidth}            

            \begin{figure}

                \caption{$h(x,y) \defeq (x \cdot y) + (\overline{x} + y)$}
            
                \begin{tikzpicture}
                    [scale=0.5,auto=left,
                        every node/.style={circle, minimum height=4mm, draw=black, thick, align=center, text depth = 0pt, transform shape},
                        every edge/.append style={draw=blue, thick}
                    ]
                    %\draw[help lines] (0,0) grid (6,6);
                    
                    \node (x1) at (3.0, 5.0) {$x$};
                    \node (y1) at (4.5, 3.0) {$y$};
                    %\node[rectangle, scale=0.7] (0a) at (1.5, 1.0) {$0$};
                    \node[rectangle, scale=0.7] (1a) at (4.5, 1.0) {$1$};

                    \node[blue] (x2) at (1.5, 1.0) {$x$};
                    \node (y2) at (3.0, -1.0) {$y$};
                    \node[rectangle, scale=0.7] (1b) at (0.0, -3.0) {$1$};
                    \node[rectangle, scale=0.7] (0) at (3.0, -3.0) {$0$};
                    
                    \draw[dashed, thick] (x1) to (x2);
                    \draw[thick] (x1) to (y1);
                    \draw[thick] (y1) to (1a);
                    \draw[dashed, thick] (y1) to (x2);

                    \draw[dashed, thick] (x2) to (1b);
                    \draw[thick] (x2) to (y2);
                    \draw[dashed, thick] (y2) to (0);
                    \draw[thick] (y2) to (1b);
                    
                \end{tikzpicture}

            \end{figure}
    \end{columns}
    
\end{frame}
%-------------------------------------------

%-------------------------------------------
\begin{frame}{Forma ``inocente'' de construir \uppercase{BDD}s}

    \begin{outline}[enumerate]
        \1 Para cada variável booleana em uma função, um BDD de variável ($B_{x_i}$) é criado
        
        \vspace{1em}
        
        \1 Tais BDDs são então unidos conforme as operações booleanas constantes na função
        
        \vspace{1em}
        
        \1 Por fim, o BDD resultante é reduzido com as simplificações C1-C3
    \end{outline}

\end{frame}
%-------------------------------------------

%-------------------------------------------
\begin{frame}{Exemplo: $(x \cdot \overline{y}) + (\overline{x} \cdot y)$}

    Passo 1: criação de $B_{x_i}$

    \vspace{-1em}
    
    \begin{columns}[c]
        \column{0.5\textwidth}
        
            \begin{figure}

                \caption{$B_x$}
            
                \begin{tikzpicture}
                    [scale=0.7,auto=left,
                        every node/.style={circle, minimum height=4mm, draw=black, thick, align=center, text depth = 0pt, transform shape},
                        every edge/.append style={draw=blue, thick}
                    ]
                    \node (x) at (1.0, 2.0) {$x$};
                    \node[rectangle, scale=0.7] (0) at (0, 0.0) {$0$};
                    \node[rectangle, scale=0.7] (1) at (2, 0.0) {$1$};
                    \draw[dashed, thick] (x) to (0);
                    \draw[thick] (x) to (1);
                  
                \end{tikzpicture}

            \end{figure}
            
            \vspace{-2.0em}
            
            \begin{figure}

                \caption{$B_{\overline{x}}$}
            
                \begin{tikzpicture}
                    [scale=0.7,auto=left,
                        every node/.style={circle, minimum height=4mm, draw=black, thick, align=center, text depth = 0pt, transform shape},
                        every edge/.append style={draw=blue, thick}
                    ]
                    \node (x) at (1.0, 2.0) {$x$};
                    \node[rectangle, scale=0.7] (1) at (0, 0.0) {$1$};
                    \node[rectangle, scale=0.7] (0) at (2, 0.0) {$0$};
                    \draw[dashed, thick] (x) to (1);
                    \draw[thick] (x) to (0);
                  
                \end{tikzpicture}

            \end{figure}
            
        \column{0.5\textwidth}            

            \begin{figure}

                \caption{$B_{\overline{y}}$}
            
                \begin{tikzpicture}
                    [scale=0.7,auto=left,
                        every node/.style={circle, minimum height=4mm, draw=black, thick, align=center, text depth = 0pt, transform shape},
                        every edge/.append style={draw=blue, thick}
                    ]
                    \node (y) at (1.0, 2.0) {$y$};
                    \node[rectangle, scale=0.7] (1) at (0, 0.0) {$1$};
                    \node[rectangle, scale=0.7] (0) at (2, 0.0) {$0$};
                    \draw[dashed, thick] (y) to (1);
                    \draw[thick] (y) to (0);
                  
                \end{tikzpicture}

            \end{figure}

            \vspace{-2.0em}
            
            \begin{figure}

                \caption{$B_y$}
            
                \begin{tikzpicture}
                    [scale=0.7,auto=left,
                        every node/.style={circle, minimum height=4mm, draw=black, thick, align=center, text depth = 0pt, transform shape},
                        every edge/.append style={draw=blue, thick}
                    ]
                    \node (y) at (1.0, 2.0) {$y$};
                    \node[rectangle, scale=0.7] (0) at (0, 0.0) {$0$};
                    \node[rectangle, scale=0.7] (1) at (2, 0.0) {$1$};
                    \draw[dashed, thick] (y) to (0);
                    \draw[thick] (y) to (1);
                  
                \end{tikzpicture}

            \end{figure}                
    \end{columns}

\end{frame}
%-------------------------------------------

%-------------------------------------------
\begin{frame}{Exemplo: $(x \cdot \overline{y}) + (\overline{x} \cdot y)$}

    Passo 2a: união dos BDDs conforme as operações
    
    \begin{columns}[c]
        \column{0.5\textwidth}
        
            \begin{figure}

                \caption{$B_x \cdot B_{\overline{y}}$}
            
                \begin{tikzpicture}
                    [scale=0.7,auto=left,
                        every node/.style={circle, minimum height=4mm, draw=black, thick, align=center, text depth = 0pt, transform shape},
                        every edge/.append style={draw=blue, thick}
                    ]
                    \node (x) at (1.0, 2.0) {$x$};
                    \node[rectangle, scale=0.7] (0a) at (0, 0.0) {$0$};

                    \node (y) at (2.0, 0.0) {$y$};
                    \node[rectangle, scale=0.7] (1) at (1, -2.0) {$1$};
                    \node[rectangle, scale=0.7] (0b) at (3, -2.0) {$0$};
                    
                    \draw[dashed, thick] (x) to (0a);
                    \draw[thick] (x) to (y);
                    \draw[dashed, thick] (y) to (1);
                    \draw[thick] (y) to (0b);
                  
                \end{tikzpicture}

            \end{figure}
            
        \column{0.5\textwidth}            

            \begin{figure}

                \caption{$B_{\overline{x}} \cdot B_y$}
            
                \begin{tikzpicture}
                    [scale=0.7,auto=left,
                        every node/.style={circle, minimum height=4mm, draw=black, thick, align=center, text depth = 0pt, transform shape},
                        every edge/.append style={draw=blue, thick}
                    ]
                    \node (x) at (1.0, 2.0) {$x$};
                    \node (y) at (0.0, 0.0) {$y$};
                    \node[rectangle, scale=0.7] (0b) at (-1, -2.0) {$0$};
                    \node[rectangle, scale=0.7] (1) at (1, -2.0) {$1$};
                    \node[rectangle, scale=0.7] (0a) at (2, 0.0) {$0$};
                    
                    \draw[dashed, thick] (x) to (y);
                    \draw[thick] (x) to (0a);
                    \draw[dashed, thick] (y) to (0b);
                    \draw[thick] (y) to (1);
                  
                \end{tikzpicture}

            \end{figure}        

    \end{columns}

\end{frame}
%-------------------------------------------

%-------------------------------------------
\begin{frame}{Exemplo: $(x \cdot \overline{y}) + (\overline{x} \cdot y)$}

    Passo 2b: união dos BDDs conforme as operações
    
    \vspace{-0.5em}
    
    \begin{figure}

        \caption{\textcolor{magenta}{$(B_x \cdot B_{\overline{y}})$} $+$ \textcolor{Maroon}{$(B_{\overline{x}} \cdot B_y)$}}
    
        \vspace{-1em}
    
        \begin{tikzpicture}
            [scale=0.6,auto=left,
                every node/.style={circle, minimum height=4mm, draw=black, thick, align=center, text depth = 0pt, transform shape},
                every edge/.append style={draw=blue, thick}
            ]
            %\draw[help lines] (0,0) grid (12,8);
                    
            \node[magenta] (x1) at (6.0, 8.0) {$x$};
            \node[Maroon] (x2) at (4.0, 6.0) {$x$};
            \node[magenta] (y1) at (8.0, 6.0) {$y$};
            \node[Maroon] (y2) at (3.0, 4.0) {$y$};
            \node[rectangle, scale=0.7, Maroon] (0a) at (5.0, 4.0) {$0$};
            \node[rectangle, scale=0.7, magenta] (1a) at (7.0, 4.0) {$1$};
            \node[Maroon] (x3) at (9.0, 4.0) {$x$};
            \node[rectangle, scale=0.7, Maroon] (0b) at (2.0, 2.0) {$0$};
            \node[rectangle, scale=0.7, Maroon] (1b) at (4.0, 2.0) {$1$};
            \node[Maroon] (y3) at (8.0, 2.0) {$y$};
            \node[rectangle, scale=0.7, Maroon] (0c) at (10.0, 2.0) {$0$};
            \node[rectangle, scale=0.7, Maroon] (0d) at (7.0, 0.0) {$0$};
            \node[rectangle, scale=0.7, Maroon] (1c) at (9.0, 0.0) {$1$};
            
            \draw[dashed, thick, magenta] (x1) to (x2);
            \draw[thick, magenta] (x1) to (y1);
            \draw[dashed, thick, Maroon] (x2) to (y2);
            \draw[thick, Maroon] (x2) to (0a);
            \draw[dashed, thick, magenta] (y1) to (1a);
            \draw[thick, magenta] (y1) to (x3);
            \draw[dashed, thick, Maroon] (y2) to (0b);
            \draw[thick, Maroon] (y2) to (1b);
            \draw[dashed, thick, Maroon] (x3) to (y3);
            \draw[thick, Maroon] (x3) to (0c);
            \draw[dashed, thick, Maroon] (y3) to (0d);
            \draw[thick, Maroon] (y3) to (1c);
            
        \end{tikzpicture}

    \end{figure}

\end{frame}
%-------------------------------------------

%-------------------------------------------
\begin{frame}{Exemplo: $(x \cdot \overline{y}) + (\overline{x} \cdot y)$}

    Passo 3a: redução do BDD gerado
    
    \begin{figure}

        \begin{tikzpicture}
            [scale=0.6,auto=left,
                every node/.style={circle, minimum height=4mm, draw=black, thick, align=center, text depth = 0pt, transform shape},
                every edge/.append style={draw=blue, thick}
            ]
            %\draw[help lines] (0,0) grid (19,8);
                    
            \node (x1) at (4.0, 8.0) {$x$};
            \node (x2) at (2.0, 6.0) {$x$};
            \node (y1) at (6.0, 6.0) {$y$};
            \node (y2) at (1.0, 4.0) {$y$};
            \node[rectangle, scale=0.7, blue] (0a) at (3.0, 4.0) {$0$};
            \node[rectangle, scale=0.7, blue] (1a) at (5.0, 4.0) {$1$};
            \node (x3) at (7.0, 4.0) {$x$};
            \node[rectangle, scale=0.7, red] (0b) at (0.0, 2.0) {$0$};
            \node[rectangle, scale=0.7, red] (1b) at (2.0, 2.0) {$1$};
            \node (y3) at (6.0, 2.0) {$y$};
            \node[rectangle, scale=0.7, red] (0c) at (8.0, 2.0) {$0$};
            \node[rectangle, scale=0.7, red] (0d) at (5.0, 0.0) {$0$};
            \node[rectangle, scale=0.7, red] (1c) at (7.0, 0.0) {$1$};
            
            \draw[dashed, thick] (x1) to (x2);
            \draw[thick] (x1) to (y1);
            \draw[dashed, thick] (x2) to (y2);
            \draw[thick] (x2) to (0a);
            \draw[dashed, thick] (y1) to (1a);
            \draw[thick] (y1) to (x3);
            \draw[dashed, thick, blue] (y2) to (0b);
            \draw[thick, blue] (y2) to (1b);
            \draw[dashed, thick] (x3) to (y3);
            \draw[thick, blue] (x3) to (0c);
            \draw[dashed, thick, blue] (y3) to (0d);
            \draw[thick, blue] (y3) to (1c);
            
            % Arrow with text
            \node[fill=white, single arrow, draw=black] at (9.0,6.0) {\scriptsize C1};

            % Rightmost graph
            \node (x1_) at (14.0, 8.0) {$x$};
            \node (x2_) at (12.0, 6.0) {$x$};
            \node (y1_) at (16.0, 6.0) {$y$};
            \node (y2_) at (11.0, 4.0) {$y$};
            \node (x3_) at (17.0, 4.0) {$x$};
            \node (y3_) at (16.0, 2.0) {$y$};
            \node[rectangle, scale=0.7, blue] (0_) at (12.5, 0.0) {$0$};
            \node[rectangle, scale=0.7, blue] (1_) at (15.5, 0.0) {$1$};
            
            \draw[dashed, thick] (x1_) to (x2_);
            \draw[thick] (x1_) to (y1_);
            \draw[dashed, thick] (x2_) to (y2_);
            \draw[thick] (x2_) to (0_);
            \draw[dashed, thick, bend right] (y1_) to (1_);
            \draw[thick] (y1_) to (x3_);
            \draw[dashed, thick, bend right, blue] (y2_) to (0_);
            \draw[thick, blue] (y2_) to (1_);
            \draw[dashed, thick] (x3_) to (y3_);
            \draw[thick, bend right, blue] (x3_) to (0_);
            \draw[dashed, thick, blue] (y3_) to (0_);
            \draw[thick, blue] (y3_) to (1_);
            
        \end{tikzpicture}

    \end{figure}

\end{frame}
%-------------------------------------------

%-------------------------------------------
\begin{frame}{Exemplo: $(x \cdot \overline{y}) + (\overline{x} \cdot y)$}

    Passo 3b: redução do BDD gerado
    
    \begin{figure}

        \begin{tikzpicture}
            [scale=0.6,auto=left,
                every node/.style={circle, minimum height=4mm, draw=black, thick, align=center, text depth = 0pt, transform shape},
                every edge/.append style={draw=blue, thick}
            ]
            %\draw[help lines] (0,0) grid (19,8);
            
            \node (x1) at (4.0, 8.0) {$x$};
            \node (x2) at (2.0, 6.0) {$x$};
            \node (y1) at (6.0, 6.0) {$y$};
            \node[blue] (y2) at (1.0, 4.0) {$y$};
            \node (x3) at (7.0, 4.0) {$x$};
            \node[red] (y3) at (6.0, 2.0) {$y$};
            \node[rectangle, scale=0.7] (0) at (2.5, 0.0) {$0$};
            \node[rectangle, scale=0.7] (1) at (5.5, 0.0) {$1$};
            
            \draw[dashed, thick] (x1) to (x2);
            \draw[thick] (x1) to (y1);
            \draw[dashed, thick] (x2) to (y2);
            \draw[thick] (x2) to (0);
            \draw[dashed, thick, bend right] (y1) to (1);
            \draw[thick] (y1) to (x3);
            \draw[dashed, thick, bend right, blue] (y2) to (0);
            \draw[thick, blue] (y2) to (1);
            \draw[dashed, thick, blue] (x3) to (y3);
            \draw[thick, bend right] (x3) to (0);
            \draw[dashed, thick, red] (y3) to (0);
            \draw[thick, red] (y3) to (1);
            
            % Arrow with text
            \node[fill=white, single arrow, draw=black] at (9.0,6.0) {\scriptsize C3};

            % Rightmost graph
            \node (x1_) at (14.0, 8.0) {$x$};
            \node (x2_) at (12.0, 6.0) {$x$};
            \node (y1_) at (16.0, 6.0) {$y$};
            \node[blue] (y2_) at (11.0, 2.0) {$y$};
            \node (x3_) at (17.0, 4.0) {$x$};
            \node[rectangle, scale=0.7] (0_) at (13.0, 0.0) {$0$};
            \node[rectangle, scale=0.7] (1_) at (16.0, 0.0) {$1$};
            
            \draw[dashed, thick] (x1_) to (x2_);
            \draw[thick] (x1_) to (y1_);
            \draw[dashed, thick] (x2_) to (y2_);
            \draw[thick] (x2_) to (0_);
            \draw[dashed, thick, bend right] (y1_) to (1_);
            \draw[thick] (y1_) to (x3_);
            \draw[dashed, thick, bend right, blue] (y2_) to (0_);
            \draw[thick, blue] (y2_) to (1_);
            \draw[dashed, thick, blue, bend right] (x3_) to (y2_);
            \draw[thick] (x3_) to (0_);
            
        \end{tikzpicture}

    \end{figure}

\end{frame}
%-------------------------------------------

%-------------------------------------------
\begin{frame}{Exemplo: $(x \cdot \overline{y}) + (\overline{x} \cdot y)$}

    Passo 3c: redução do BDD gerado
    
    \begin{figure}

        \begin{tikzpicture}
            [scale=0.6,auto=left,
                every node/.style={circle, minimum height=4mm, draw=black, thick, align=center, text depth = 0pt, transform shape},
                every edge/.append style={draw=blue, thick}
            ]
            %\draw[help lines] (0,0) grid (19,8);
            
            \node (x1) at (4.0, 8.0) {$x$};
            \node[blue] (x2) at (2.0, 6.0) {$x$};
            \node (y1) at (6.0, 6.0) {$y$};
            \node (y2) at (1.0, 2.0) {$y$};
            \node[red] (x3) at (7.0, 4.0) {$x$};
            \node[rectangle, scale=0.7] (0) at (3.0, 0.0) {$0$};
            \node[rectangle, scale=0.7] (1) at (6.0, 0.0) {$1$};
            
            \draw[dashed, thick] (x1) to (x2);
            \draw[thick] (x1) to (y1);
            \draw[dashed, thick, blue] (x2) to (y2);
            \draw[thick, blue] (x2) to (0);
            \draw[dashed, thick, bend right] (y1) to (1);
            \draw[thick, blue] (y1) to (x3);
            \draw[dashed, thick, bend right] (y2) to (0);
            \draw[thick] (y2) to (1);
            \draw[dashed, thick, bend right, red] (x3) to (y2);
            \draw[thick, red] (x3) to (0);
            
            % Arrow with text
            \node[fill=white, single arrow, draw=black] at (9.0,6.0) {\scriptsize C3};

            % Rightmost graph
            \node (x1_) at (14.0, 8.0) {$x$};
            \node[blue] (x2_) at (12.0, 5.0) {$x$};
            \node (y1_) at (16.0, 6.0) {$y$};
            \node (y2_) at (11.0, 2.0) {$y$};
            \node[rectangle, scale=0.7] (0_) at (13.0, 0.0) {$0$};
            \node[rectangle, scale=0.7] (1_) at (16.0, 0.0) {$1$};
            
            \draw[dashed, thick] (x1_) to (x2_);
            \draw[thick] (x1_) to (y1_);
            \draw[dashed, thick, blue] (x2_) to (y2_);
            \draw[thick, blue] (x2_) to (0_);
            \draw[dashed, thick] (y1_) to (1_);
            \draw[thick, blue] (y1_) to (x2_);
            \draw[dashed, thick] (y2_) to (0_);
            \draw[thick] (y2_) to (1_);
        \end{tikzpicture}

    \end{figure}

\end{frame}
%-------------------------------------------

%-------------------------------------------
\begin{frame}{Comparação com a tabela-verdade}
    \centering
    $f(x,y) \defeq (x \cdot \overline{y}) + (\overline{x} \cdot y)$

    \begin{columns}[c]
        \column{0.4\textwidth}

            \begin{figure}
            
                \begin{tikzpicture}
                    [scale=0.8,auto=left,
                        every node/.style={circle, minimum height=4mm, draw=black, thick, align=center, text depth = 0pt, transform shape},
                        every edge/.append style={draw=blue, thick}
                    ]
                    %\draw[help lines] (0,0) grid (8,6);

                    \node (x1) at (4.0, 6.0) {$x$};
                    \node (x2) at (2.0, 3.0) {$x$};
                    \node (y1) at (6.0, 4.0) {$y$};
                    \node (y2) at (4.0, 2.0) {$y$};
                    \node[rectangle, scale=0.7] (0) at (3.0, 0.0) {$0$};
                    \node[rectangle, scale=0.7] (1) at (6.0, 0.0) {$1$};
                    
                    \draw[dashed, thick] (x1) to (x2);
                    \draw[thick] (x1) to (y1);
                    \draw[dashed, thick] (x2) to (y2);
                    \draw[thick] (x2) to (0);
                    \draw[dashed, thick] (y1) to (1);
                    \draw[thick] (y1) to (x2);
                    \draw[dashed, thick] (y2) to (0);
                    \draw[thick] (y2) to (1);                    

                \end{tikzpicture}

            \end{figure}

        \column{0.6\textwidth}
            
            \begin{center}
                \begin{table}
                    \begin{tabular}{cc|c}
                        $x$ & $y$ & $f(x,y)$\\
                        \hline
                        $0$ & $0$ & $0$\\
                        $0$ & $1$ & $1$\\
                        $1$ & $0$ & $1$\\
                        $1$ & $1$ & $0$\\
                    \end{tabular}
                \end{table}
            \end{center}


    \end{columns}
    
\end{frame}
%-------------------------------------------

%-------------------------------------------
\begin{frame}{Múltiplas ocorrências de mesma variável}

    \begin{columns}[c]
        \column{0.4\textwidth}

            \begin{figure}
            
                \begin{tikzpicture}
                    [scale=0.8,auto=left,
                        every node/.style={circle, minimum height=4mm, draw=black, thick, align=center, text depth = 0pt, transform shape},
                        every edge/.append style={draw=blue, thick}
                    ]
                    %\draw[help lines] (0,0) grid (8,6);

                    \node (x1) at (4.0, 6.0) {$x$};
                    \node (y1) at (2.5, 4.5) {$y$};
                    \node (z1) at (5.5, 4.5) {$z$};
                    \node (x2) at (1.5, 3.0) {$x$};
                    \node (x3) at (6.5, 3.0) {$x$};
                    \node (y2) at (4.3, 3.0) {$y$};
                    
                    \node[rectangle, scale=0.7] (0) at (2.0, 1.0) {$0$};
                    \node[rectangle, scale=0.7] (1) at (6.0, 1.0) {$1$};

                    \draw[dashed, thick] (x1) to (y1);
                    \draw[thick] (x1) to (z1);
                    \draw[dashed, thick] (y1) to (x2);
                    \draw[thick, bend right=15] (y1) to (1);
                    \draw[thick, magenta] (x2) to (1);
                    \draw[dashed, thick] (x2) to (0);
                    \draw[thick] (x3) to (1);
                    \draw[dashed, thick] (x3) to (0);
                    \draw[dashed, thick] (z1) to (y2);
                    \draw[thick] (z1) to (x3);
                    \draw[dashed, thick] (y2) to (0);
                    \draw[thick] (y2) to (1);                    

                \end{tikzpicture}

            \end{figure}

        \column{0.6\textwidth}
            
            \begin{outline}
                \small
                \1 A definição não impede uma variável de ocorrer mais de uma vez em um caminho
                
                \vspace{1em}
                
                \1 Mas tal representação pode incorrer em desperdícios
                    \2[-] {\scriptsize linha sólida do $x$ à esquerda (colorida) jamais será percorrida}
            \end{outline}
            
            \begin{center}
                \ovalbox{
                    \begin{minipage}[h]{0.9\columnwidth}
                        \noindent\scriptsize\justifying
                        Comum após as operações de conjunção e disjunção discutidas anteriormente (algoritmos melhores serão discutidos à frente)
                    \end{minipage}
                }
            \end{center}            

    \end{columns}
    
\end{frame}
%-------------------------------------------

%-------------------------------------------
\begin{frame}{Comparação de \uppercase{BDD}s}

    {\small Além de tornar um BDD ineficiente, ocorrências múltiplas de uma variável também dificultam a comparação de BDDs}

    \begin{columns}[c]
        \column{0.5\textwidth}

            \begin{figure}
            
                \begin{tikzpicture}
                    [scale=0.8,auto=left,
                        every node/.style={circle, minimum height=4mm, draw=black, thick, align=center, text depth = 0pt, transform shape},
                        every edge/.append style={draw=blue, thick}
                    ]
                    %\draw[help lines] (0,0) grid (8,6);
                  
                    \node (x1) at (4.0, 6.0) {$x$};
                    \node (y1) at (2.5, 4.5) {$y$};
                    \node (z1) at (5.5, 4.5) {$z$};
                    \node (x2) at (1.5, 3.0) {$x$};
                    \node (x3) at (6.5, 3.0) {$x$};
                    \node (y2) at (4.3, 3.0) {$y$};
                    
                    \node[rectangle, scale=0.7] (0) at (2.0, 1.0) {$0$};
                    \node[rectangle, scale=0.7] (1) at (6.0, 1.0) {$1$};

                    \draw[dashed, thick] (x1) to (y1);
                    \draw[thick] (x1) to (z1);
                    \draw[dashed, thick] (y1) to (x2);
                    \draw[thick, bend right=15] (y1) to (1);
                    \draw[thick] (x2) to (1);
                    \draw[dashed, thick] (x2) to (0);
                    \draw[thick] (x3) to (1);
                    \draw[dashed, thick] (x3) to (0);
                    \draw[dashed, thick] (z1) to (y2);
                    \draw[thick] (z1) to (x3);
                    \draw[dashed, thick] (y2) to (0);
                    \draw[thick] (y2) to (1);

                \end{tikzpicture}

            \end{figure}

        \column{0.5\textwidth}
            
            \begin{figure}
            
                \begin{tikzpicture}
                    [scale=0.8,auto=left,
                        every node/.style={circle, minimum height=4mm, draw=black, thick, align=center, text depth = 0pt, transform shape},
                        every edge/.append style={draw=blue, thick}
                    ]
                    %\draw[help lines] (0,0) grid (8,6);
                  
                    \node (x) at (4.0, 6.0) {$x$};
                    \node (y1) at (2.5, 4.5) {$y$};
                    \node (y2) at (5.5, 4.5) {$y$};
                    \node (z) at (4.0, 3.0) {$z$};
                    
                    \node[rectangle, scale=0.7] (0) at (2.5, 1.0) {$0$};
                    \node[rectangle, scale=0.7] (1) at (5.5, 1.0) {$1$};

                    \draw[dashed, thick] (x) to (y1);
                    \draw[thick] (x) to (y2);
                    \draw[dashed, thick] (y1) to (0);
                    \draw[thick, bend right] (y1) to (1);
                    \draw[dashed, thick] (y2) to (z);
                    \draw[thick] (y2) to (1);
                    \draw[dashed, thick] (z) to (0);
                    \draw[thick] (z) to (1);
                    
                \end{tikzpicture}

            \end{figure}

    \end{columns}
    
\end{frame}
%-------------------------------------------

\end{document}